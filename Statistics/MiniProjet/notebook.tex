
% Default to the notebook output style

    


% Inherit from the specified cell style.




    
\documentclass[11pt]{article}

    
    
    \usepackage[T1]{fontenc}
    % Nicer default font (+ math font) than Computer Modern for most use cases
    \usepackage{mathpazo}

    % Basic figure setup, for now with no caption control since it's done
    % automatically by Pandoc (which extracts ![](path) syntax from Markdown).
    \usepackage{graphicx}
    % We will generate all images so they have a width \maxwidth. This means
    % that they will get their normal width if they fit onto the page, but
    % are scaled down if they would overflow the margins.
    \makeatletter
    \def\maxwidth{\ifdim\Gin@nat@width>\linewidth\linewidth
    \else\Gin@nat@width\fi}
    \makeatother
    \let\Oldincludegraphics\includegraphics
    % Set max figure width to be 80% of text width, for now hardcoded.
    \renewcommand{\includegraphics}[1]{\Oldincludegraphics[width=.8\maxwidth]{#1}}
    % Ensure that by default, figures have no caption (until we provide a
    % proper Figure object with a Caption API and a way to capture that
    % in the conversion process - todo).
    \usepackage{caption}
    \DeclareCaptionLabelFormat{nolabel}{}
    \captionsetup{labelformat=nolabel}

    \usepackage{adjustbox} % Used to constrain images to a maximum size 
    \usepackage{xcolor} % Allow colors to be defined
    \usepackage{enumerate} % Needed for markdown enumerations to work
    \usepackage{geometry} % Used to adjust the document margins
    \usepackage{amsmath} % Equations
    \usepackage{amssymb} % Equations
    \usepackage{textcomp} % defines textquotesingle
    % Hack from http://tex.stackexchange.com/a/47451/13684:
    \AtBeginDocument{%
        \def\PYZsq{\textquotesingle}% Upright quotes in Pygmentized code
    }
    \usepackage{upquote} % Upright quotes for verbatim code
    \usepackage{eurosym} % defines \euro
    \usepackage[mathletters]{ucs} % Extended unicode (utf-8) support
    \usepackage[utf8x]{inputenc} % Allow utf-8 characters in the tex document
    \usepackage{fancyvrb} % verbatim replacement that allows latex
    \usepackage{grffile} % extends the file name processing of package graphics 
                         % to support a larger range 
    % The hyperref package gives us a pdf with properly built
    % internal navigation ('pdf bookmarks' for the table of contents,
    % internal cross-reference links, web links for URLs, etc.)
    \usepackage{hyperref}
    \usepackage{longtable} % longtable support required by pandoc >1.10
    \usepackage{booktabs}  % table support for pandoc > 1.12.2
    \usepackage[inline]{enumitem} % IRkernel/repr support (it uses the enumerate* environment)
    \usepackage[normalem]{ulem} % ulem is needed to support strikethroughs (\sout)
                                % normalem makes italics be italics, not underlines
    

    
    
    % Colors for the hyperref package
    \definecolor{urlcolor}{rgb}{0,.145,.698}
    \definecolor{linkcolor}{rgb}{.71,0.21,0.01}
    \definecolor{citecolor}{rgb}{.12,.54,.11}

    % ANSI colors
    \definecolor{ansi-black}{HTML}{3E424D}
    \definecolor{ansi-black-intense}{HTML}{282C36}
    \definecolor{ansi-red}{HTML}{E75C58}
    \definecolor{ansi-red-intense}{HTML}{B22B31}
    \definecolor{ansi-green}{HTML}{00A250}
    \definecolor{ansi-green-intense}{HTML}{007427}
    \definecolor{ansi-yellow}{HTML}{DDB62B}
    \definecolor{ansi-yellow-intense}{HTML}{B27D12}
    \definecolor{ansi-blue}{HTML}{208FFB}
    \definecolor{ansi-blue-intense}{HTML}{0065CA}
    \definecolor{ansi-magenta}{HTML}{D160C4}
    \definecolor{ansi-magenta-intense}{HTML}{A03196}
    \definecolor{ansi-cyan}{HTML}{60C6C8}
    \definecolor{ansi-cyan-intense}{HTML}{258F8F}
    \definecolor{ansi-white}{HTML}{C5C1B4}
    \definecolor{ansi-white-intense}{HTML}{A1A6B2}

    % commands and environments needed by pandoc snippets
    % extracted from the output of `pandoc -s`
    \providecommand{\tightlist}{%
      \setlength{\itemsep}{0pt}\setlength{\parskip}{0pt}}
    \DefineVerbatimEnvironment{Highlighting}{Verbatim}{commandchars=\\\{\}}
    % Add ',fontsize=\small' for more characters per line
    \newenvironment{Shaded}{}{}
    \newcommand{\KeywordTok}[1]{\textcolor[rgb]{0.00,0.44,0.13}{\textbf{{#1}}}}
    \newcommand{\DataTypeTok}[1]{\textcolor[rgb]{0.56,0.13,0.00}{{#1}}}
    \newcommand{\DecValTok}[1]{\textcolor[rgb]{0.25,0.63,0.44}{{#1}}}
    \newcommand{\BaseNTok}[1]{\textcolor[rgb]{0.25,0.63,0.44}{{#1}}}
    \newcommand{\FloatTok}[1]{\textcolor[rgb]{0.25,0.63,0.44}{{#1}}}
    \newcommand{\CharTok}[1]{\textcolor[rgb]{0.25,0.44,0.63}{{#1}}}
    \newcommand{\StringTok}[1]{\textcolor[rgb]{0.25,0.44,0.63}{{#1}}}
    \newcommand{\CommentTok}[1]{\textcolor[rgb]{0.38,0.63,0.69}{\textit{{#1}}}}
    \newcommand{\OtherTok}[1]{\textcolor[rgb]{0.00,0.44,0.13}{{#1}}}
    \newcommand{\AlertTok}[1]{\textcolor[rgb]{1.00,0.00,0.00}{\textbf{{#1}}}}
    \newcommand{\FunctionTok}[1]{\textcolor[rgb]{0.02,0.16,0.49}{{#1}}}
    \newcommand{\RegionMarkerTok}[1]{{#1}}
    \newcommand{\ErrorTok}[1]{\textcolor[rgb]{1.00,0.00,0.00}{\textbf{{#1}}}}
    \newcommand{\NormalTok}[1]{{#1}}
    
    % Additional commands for more recent versions of Pandoc
    \newcommand{\ConstantTok}[1]{\textcolor[rgb]{0.53,0.00,0.00}{{#1}}}
    \newcommand{\SpecialCharTok}[1]{\textcolor[rgb]{0.25,0.44,0.63}{{#1}}}
    \newcommand{\VerbatimStringTok}[1]{\textcolor[rgb]{0.25,0.44,0.63}{{#1}}}
    \newcommand{\SpecialStringTok}[1]{\textcolor[rgb]{0.73,0.40,0.53}{{#1}}}
    \newcommand{\ImportTok}[1]{{#1}}
    \newcommand{\DocumentationTok}[1]{\textcolor[rgb]{0.73,0.13,0.13}{\textit{{#1}}}}
    \newcommand{\AnnotationTok}[1]{\textcolor[rgb]{0.38,0.63,0.69}{\textbf{\textit{{#1}}}}}
    \newcommand{\CommentVarTok}[1]{\textcolor[rgb]{0.38,0.63,0.69}{\textbf{\textit{{#1}}}}}
    \newcommand{\VariableTok}[1]{\textcolor[rgb]{0.10,0.09,0.49}{{#1}}}
    \newcommand{\ControlFlowTok}[1]{\textcolor[rgb]{0.00,0.44,0.13}{\textbf{{#1}}}}
    \newcommand{\OperatorTok}[1]{\textcolor[rgb]{0.40,0.40,0.40}{{#1}}}
    \newcommand{\BuiltInTok}[1]{{#1}}
    \newcommand{\ExtensionTok}[1]{{#1}}
    \newcommand{\PreprocessorTok}[1]{\textcolor[rgb]{0.74,0.48,0.00}{{#1}}}
    \newcommand{\AttributeTok}[1]{\textcolor[rgb]{0.49,0.56,0.16}{{#1}}}
    \newcommand{\InformationTok}[1]{\textcolor[rgb]{0.38,0.63,0.69}{\textbf{\textit{{#1}}}}}
    \newcommand{\WarningTok}[1]{\textcolor[rgb]{0.38,0.63,0.69}{\textbf{\textit{{#1}}}}}
    
    
    % Define a nice break command that doesn't care if a line doesn't already
    % exist.
    \def\br{\hspace*{\fill} \\* }
    % Math Jax compatability definitions
    \def\gt{>}
    \def\lt{<}
    % Document parameters
    \title{Mini-Projet\_BourhanDernayka}
    
    
    

    % Pygments definitions
    
\makeatletter
\def\PY@reset{\let\PY@it=\relax \let\PY@bf=\relax%
    \let\PY@ul=\relax \let\PY@tc=\relax%
    \let\PY@bc=\relax \let\PY@ff=\relax}
\def\PY@tok#1{\csname PY@tok@#1\endcsname}
\def\PY@toks#1+{\ifx\relax#1\empty\else%
    \PY@tok{#1}\expandafter\PY@toks\fi}
\def\PY@do#1{\PY@bc{\PY@tc{\PY@ul{%
    \PY@it{\PY@bf{\PY@ff{#1}}}}}}}
\def\PY#1#2{\PY@reset\PY@toks#1+\relax+\PY@do{#2}}

\expandafter\def\csname PY@tok@w\endcsname{\def\PY@tc##1{\textcolor[rgb]{0.73,0.73,0.73}{##1}}}
\expandafter\def\csname PY@tok@c\endcsname{\let\PY@it=\textit\def\PY@tc##1{\textcolor[rgb]{0.25,0.50,0.50}{##1}}}
\expandafter\def\csname PY@tok@cp\endcsname{\def\PY@tc##1{\textcolor[rgb]{0.74,0.48,0.00}{##1}}}
\expandafter\def\csname PY@tok@k\endcsname{\let\PY@bf=\textbf\def\PY@tc##1{\textcolor[rgb]{0.00,0.50,0.00}{##1}}}
\expandafter\def\csname PY@tok@kp\endcsname{\def\PY@tc##1{\textcolor[rgb]{0.00,0.50,0.00}{##1}}}
\expandafter\def\csname PY@tok@kt\endcsname{\def\PY@tc##1{\textcolor[rgb]{0.69,0.00,0.25}{##1}}}
\expandafter\def\csname PY@tok@o\endcsname{\def\PY@tc##1{\textcolor[rgb]{0.40,0.40,0.40}{##1}}}
\expandafter\def\csname PY@tok@ow\endcsname{\let\PY@bf=\textbf\def\PY@tc##1{\textcolor[rgb]{0.67,0.13,1.00}{##1}}}
\expandafter\def\csname PY@tok@nb\endcsname{\def\PY@tc##1{\textcolor[rgb]{0.00,0.50,0.00}{##1}}}
\expandafter\def\csname PY@tok@nf\endcsname{\def\PY@tc##1{\textcolor[rgb]{0.00,0.00,1.00}{##1}}}
\expandafter\def\csname PY@tok@nc\endcsname{\let\PY@bf=\textbf\def\PY@tc##1{\textcolor[rgb]{0.00,0.00,1.00}{##1}}}
\expandafter\def\csname PY@tok@nn\endcsname{\let\PY@bf=\textbf\def\PY@tc##1{\textcolor[rgb]{0.00,0.00,1.00}{##1}}}
\expandafter\def\csname PY@tok@ne\endcsname{\let\PY@bf=\textbf\def\PY@tc##1{\textcolor[rgb]{0.82,0.25,0.23}{##1}}}
\expandafter\def\csname PY@tok@nv\endcsname{\def\PY@tc##1{\textcolor[rgb]{0.10,0.09,0.49}{##1}}}
\expandafter\def\csname PY@tok@no\endcsname{\def\PY@tc##1{\textcolor[rgb]{0.53,0.00,0.00}{##1}}}
\expandafter\def\csname PY@tok@nl\endcsname{\def\PY@tc##1{\textcolor[rgb]{0.63,0.63,0.00}{##1}}}
\expandafter\def\csname PY@tok@ni\endcsname{\let\PY@bf=\textbf\def\PY@tc##1{\textcolor[rgb]{0.60,0.60,0.60}{##1}}}
\expandafter\def\csname PY@tok@na\endcsname{\def\PY@tc##1{\textcolor[rgb]{0.49,0.56,0.16}{##1}}}
\expandafter\def\csname PY@tok@nt\endcsname{\let\PY@bf=\textbf\def\PY@tc##1{\textcolor[rgb]{0.00,0.50,0.00}{##1}}}
\expandafter\def\csname PY@tok@nd\endcsname{\def\PY@tc##1{\textcolor[rgb]{0.67,0.13,1.00}{##1}}}
\expandafter\def\csname PY@tok@s\endcsname{\def\PY@tc##1{\textcolor[rgb]{0.73,0.13,0.13}{##1}}}
\expandafter\def\csname PY@tok@sd\endcsname{\let\PY@it=\textit\def\PY@tc##1{\textcolor[rgb]{0.73,0.13,0.13}{##1}}}
\expandafter\def\csname PY@tok@si\endcsname{\let\PY@bf=\textbf\def\PY@tc##1{\textcolor[rgb]{0.73,0.40,0.53}{##1}}}
\expandafter\def\csname PY@tok@se\endcsname{\let\PY@bf=\textbf\def\PY@tc##1{\textcolor[rgb]{0.73,0.40,0.13}{##1}}}
\expandafter\def\csname PY@tok@sr\endcsname{\def\PY@tc##1{\textcolor[rgb]{0.73,0.40,0.53}{##1}}}
\expandafter\def\csname PY@tok@ss\endcsname{\def\PY@tc##1{\textcolor[rgb]{0.10,0.09,0.49}{##1}}}
\expandafter\def\csname PY@tok@sx\endcsname{\def\PY@tc##1{\textcolor[rgb]{0.00,0.50,0.00}{##1}}}
\expandafter\def\csname PY@tok@m\endcsname{\def\PY@tc##1{\textcolor[rgb]{0.40,0.40,0.40}{##1}}}
\expandafter\def\csname PY@tok@gh\endcsname{\let\PY@bf=\textbf\def\PY@tc##1{\textcolor[rgb]{0.00,0.00,0.50}{##1}}}
\expandafter\def\csname PY@tok@gu\endcsname{\let\PY@bf=\textbf\def\PY@tc##1{\textcolor[rgb]{0.50,0.00,0.50}{##1}}}
\expandafter\def\csname PY@tok@gd\endcsname{\def\PY@tc##1{\textcolor[rgb]{0.63,0.00,0.00}{##1}}}
\expandafter\def\csname PY@tok@gi\endcsname{\def\PY@tc##1{\textcolor[rgb]{0.00,0.63,0.00}{##1}}}
\expandafter\def\csname PY@tok@gr\endcsname{\def\PY@tc##1{\textcolor[rgb]{1.00,0.00,0.00}{##1}}}
\expandafter\def\csname PY@tok@ge\endcsname{\let\PY@it=\textit}
\expandafter\def\csname PY@tok@gs\endcsname{\let\PY@bf=\textbf}
\expandafter\def\csname PY@tok@gp\endcsname{\let\PY@bf=\textbf\def\PY@tc##1{\textcolor[rgb]{0.00,0.00,0.50}{##1}}}
\expandafter\def\csname PY@tok@go\endcsname{\def\PY@tc##1{\textcolor[rgb]{0.53,0.53,0.53}{##1}}}
\expandafter\def\csname PY@tok@gt\endcsname{\def\PY@tc##1{\textcolor[rgb]{0.00,0.27,0.87}{##1}}}
\expandafter\def\csname PY@tok@err\endcsname{\def\PY@bc##1{\setlength{\fboxsep}{0pt}\fcolorbox[rgb]{1.00,0.00,0.00}{1,1,1}{\strut ##1}}}
\expandafter\def\csname PY@tok@kc\endcsname{\let\PY@bf=\textbf\def\PY@tc##1{\textcolor[rgb]{0.00,0.50,0.00}{##1}}}
\expandafter\def\csname PY@tok@kd\endcsname{\let\PY@bf=\textbf\def\PY@tc##1{\textcolor[rgb]{0.00,0.50,0.00}{##1}}}
\expandafter\def\csname PY@tok@kn\endcsname{\let\PY@bf=\textbf\def\PY@tc##1{\textcolor[rgb]{0.00,0.50,0.00}{##1}}}
\expandafter\def\csname PY@tok@kr\endcsname{\let\PY@bf=\textbf\def\PY@tc##1{\textcolor[rgb]{0.00,0.50,0.00}{##1}}}
\expandafter\def\csname PY@tok@bp\endcsname{\def\PY@tc##1{\textcolor[rgb]{0.00,0.50,0.00}{##1}}}
\expandafter\def\csname PY@tok@fm\endcsname{\def\PY@tc##1{\textcolor[rgb]{0.00,0.00,1.00}{##1}}}
\expandafter\def\csname PY@tok@vc\endcsname{\def\PY@tc##1{\textcolor[rgb]{0.10,0.09,0.49}{##1}}}
\expandafter\def\csname PY@tok@vg\endcsname{\def\PY@tc##1{\textcolor[rgb]{0.10,0.09,0.49}{##1}}}
\expandafter\def\csname PY@tok@vi\endcsname{\def\PY@tc##1{\textcolor[rgb]{0.10,0.09,0.49}{##1}}}
\expandafter\def\csname PY@tok@vm\endcsname{\def\PY@tc##1{\textcolor[rgb]{0.10,0.09,0.49}{##1}}}
\expandafter\def\csname PY@tok@sa\endcsname{\def\PY@tc##1{\textcolor[rgb]{0.73,0.13,0.13}{##1}}}
\expandafter\def\csname PY@tok@sb\endcsname{\def\PY@tc##1{\textcolor[rgb]{0.73,0.13,0.13}{##1}}}
\expandafter\def\csname PY@tok@sc\endcsname{\def\PY@tc##1{\textcolor[rgb]{0.73,0.13,0.13}{##1}}}
\expandafter\def\csname PY@tok@dl\endcsname{\def\PY@tc##1{\textcolor[rgb]{0.73,0.13,0.13}{##1}}}
\expandafter\def\csname PY@tok@s2\endcsname{\def\PY@tc##1{\textcolor[rgb]{0.73,0.13,0.13}{##1}}}
\expandafter\def\csname PY@tok@sh\endcsname{\def\PY@tc##1{\textcolor[rgb]{0.73,0.13,0.13}{##1}}}
\expandafter\def\csname PY@tok@s1\endcsname{\def\PY@tc##1{\textcolor[rgb]{0.73,0.13,0.13}{##1}}}
\expandafter\def\csname PY@tok@mb\endcsname{\def\PY@tc##1{\textcolor[rgb]{0.40,0.40,0.40}{##1}}}
\expandafter\def\csname PY@tok@mf\endcsname{\def\PY@tc##1{\textcolor[rgb]{0.40,0.40,0.40}{##1}}}
\expandafter\def\csname PY@tok@mh\endcsname{\def\PY@tc##1{\textcolor[rgb]{0.40,0.40,0.40}{##1}}}
\expandafter\def\csname PY@tok@mi\endcsname{\def\PY@tc##1{\textcolor[rgb]{0.40,0.40,0.40}{##1}}}
\expandafter\def\csname PY@tok@il\endcsname{\def\PY@tc##1{\textcolor[rgb]{0.40,0.40,0.40}{##1}}}
\expandafter\def\csname PY@tok@mo\endcsname{\def\PY@tc##1{\textcolor[rgb]{0.40,0.40,0.40}{##1}}}
\expandafter\def\csname PY@tok@ch\endcsname{\let\PY@it=\textit\def\PY@tc##1{\textcolor[rgb]{0.25,0.50,0.50}{##1}}}
\expandafter\def\csname PY@tok@cm\endcsname{\let\PY@it=\textit\def\PY@tc##1{\textcolor[rgb]{0.25,0.50,0.50}{##1}}}
\expandafter\def\csname PY@tok@cpf\endcsname{\let\PY@it=\textit\def\PY@tc##1{\textcolor[rgb]{0.25,0.50,0.50}{##1}}}
\expandafter\def\csname PY@tok@c1\endcsname{\let\PY@it=\textit\def\PY@tc##1{\textcolor[rgb]{0.25,0.50,0.50}{##1}}}
\expandafter\def\csname PY@tok@cs\endcsname{\let\PY@it=\textit\def\PY@tc##1{\textcolor[rgb]{0.25,0.50,0.50}{##1}}}

\def\PYZbs{\char`\\}
\def\PYZus{\char`\_}
\def\PYZob{\char`\{}
\def\PYZcb{\char`\}}
\def\PYZca{\char`\^}
\def\PYZam{\char`\&}
\def\PYZlt{\char`\<}
\def\PYZgt{\char`\>}
\def\PYZsh{\char`\#}
\def\PYZpc{\char`\%}
\def\PYZdl{\char`\$}
\def\PYZhy{\char`\-}
\def\PYZsq{\char`\'}
\def\PYZdq{\char`\"}
\def\PYZti{\char`\~}
% for compatibility with earlier versions
\def\PYZat{@}
\def\PYZlb{[}
\def\PYZrb{]}
\makeatother


    % Exact colors from NB
    \definecolor{incolor}{rgb}{0.0, 0.0, 0.5}
    \definecolor{outcolor}{rgb}{0.545, 0.0, 0.0}



    
    % Prevent overflowing lines due to hard-to-break entities
    \sloppy 
    % Setup hyperref package
    \hypersetup{
      breaklinks=true,  % so long urls are correctly broken across lines
      colorlinks=true,
      urlcolor=urlcolor,
      linkcolor=linkcolor,
      citecolor=citecolor,
      }
    % Slightly bigger margins than the latex defaults
    
    \geometry{verbose,tmargin=1in,bmargin=1in,lmargin=1in,rmargin=1in}
    
    

    \begin{document}
    
    
    \maketitle
    
    

    
    \section{Mini-Projet de Statistiques}\label{mini-projet-de-statistiques}

\subsection{\texorpdfstring{\[Three\ Mile\ Island \]}{Three\textbackslash{} Mile\textbackslash{} Island }}\label{three-mile-island}

\subsubsection{\texorpdfstring{\emph{Préparé
par:}}{Préparé par:}}\label{pruxe9paruxe9-par}

\begin{verbatim}
Bourhan Dernayka
\end{verbatim}

    \begin{center}\rule{0.5\linewidth}{\linethickness}\end{center}

\subsubsection{\texorpdfstring{\[EXERCICE\ 1\]}{EXERCICE\textbackslash{} 1}}\label{exercice-1}

    \begin{center}\rule{0.5\linewidth}{\linethickness}\end{center}

\paragraph{EX 1.1 Data Acquiring}\label{ex-1.1-data-acquiring}

    \begin{Verbatim}[commandchars=\\\{\}]
{\color{incolor}In [{\color{incolor}32}]:} \PY{k+kn}{import} \PY{n+nn}{pandas} \PY{k}{as} \PY{n+nn}{pd}
         
         \PY{n}{filename}\PY{o}{=}\PY{l+s+s2}{\PYZdq{}}\PY{l+s+s2}{NuclearPowerAccidents2016}\PY{l+s+s2}{\PYZdq{}}
         \PY{n}{na\PYZus{}values} \PY{o}{=} \PY{p}{[}\PY{l+s+s1}{\PYZsq{}}\PY{l+s+s1}{\PYZsq{}}\PY{p}{]}
         \PY{n}{fields} \PY{o}{=} \PY{p}{[}\PY{l+s+s1}{\PYZsq{}}\PY{l+s+s1}{Date}\PY{l+s+s1}{\PYZsq{}}\PY{p}{,} \PY{l+s+s1}{\PYZsq{}}\PY{l+s+s1}{Cost (millions 2013US\PYZdl{})}\PY{l+s+s1}{\PYZsq{}}\PY{p}{]}
         \PY{n}{df} \PY{o}{=} \PY{n}{pd}\PY{o}{.}\PY{n}{read\PYZus{}csv}\PY{p}{(}\PY{n}{filename} \PY{o}{+} \PY{l+s+s1}{\PYZsq{}}\PY{l+s+s1}{.csv}\PY{l+s+s1}{\PYZsq{}}\PY{p}{,} \PY{n}{sep}\PY{o}{=}\PY{l+s+s1}{\PYZsq{}}\PY{l+s+s1}{,}\PY{l+s+s1}{\PYZsq{}}\PY{p}{,} \PY{n}{usecols}\PY{o}{=}\PY{n}{fields}\PY{p}{,} \PY{n}{low\PYZus{}memory} \PY{o}{=} \PY{k+kc}{False}\PY{p}{,} \PY{n}{na\PYZus{}values}\PY{o}{=}\PY{n}{na\PYZus{}values}\PY{p}{)}
         \PY{n}{df} \PY{o}{=} \PY{n}{df}\PY{o}{.}\PY{n}{dropna}\PY{p}{(}\PY{p}{)}
         \PY{n}{df}\PY{p}{[}\PY{l+s+s1}{\PYZsq{}}\PY{l+s+s1}{Date}\PY{l+s+s1}{\PYZsq{}}\PY{p}{]} \PY{o}{=} \PY{n}{pd}\PY{o}{.}\PY{n}{to\PYZus{}datetime}\PY{p}{(}\PY{n}{df}\PY{p}{[}\PY{l+s+s1}{\PYZsq{}}\PY{l+s+s1}{Date}\PY{l+s+s1}{\PYZsq{}}\PY{p}{]}\PY{p}{)} \PY{c+c1}{\PYZsh{} format = \PYZsq{}\PYZpc{}d/\PYZpc{}m/\PYZpc{}Y\PYZsq{}}
         \PY{n}{df} \PY{o}{=} \PY{n}{df}\PY{o}{.}\PY{n}{set\PYZus{}index}\PY{p}{(}\PY{p}{[}\PY{l+s+s1}{\PYZsq{}}\PY{l+s+s1}{Date}\PY{l+s+s1}{\PYZsq{}}\PY{p}{]}\PY{p}{,} \PY{n}{drop}\PY{o}{=}\PY{k+kc}{True}\PY{p}{)}
         \PY{n}{df} \PY{o}{=} \PY{n}{df}\PY{o}{.}\PY{n}{sort\PYZus{}index}\PY{p}{(}\PY{p}{)}
\end{Verbatim}


    \begin{Verbatim}[commandchars=\\\{\}]
{\color{incolor}In [{\color{incolor}33}]:} \PY{n+nb}{len}\PY{p}{(}\PY{n}{df}\PY{p}{)}
\end{Verbatim}


\begin{Verbatim}[commandchars=\\\{\}]
{\color{outcolor}Out[{\color{outcolor}33}]:} 175
\end{Verbatim}
            
    La date de l'accident est: \textgreater{} March 28, 1979

ce qui indique qu'on doit enlever tout les données enregistrées depuis
la date 1979-03-28

    \begin{Verbatim}[commandchars=\\\{\}]
{\color{incolor}In [{\color{incolor}34}]:} \PY{n}{endDate} \PY{o}{=} \PY{l+s+s2}{\PYZdq{}}\PY{l+s+s2}{1979\PYZhy{}03\PYZhy{}27}\PY{l+s+s2}{\PYZdq{}}
         \PY{n}{endDate} \PY{o}{=} \PY{n}{pd}\PY{o}{.}\PY{n}{to\PYZus{}datetime}\PY{p}{(}\PY{n}{endDate}\PY{p}{)}
         \PY{n}{df}\PY{o}{=}\PY{n}{df}\PY{p}{[}\PY{p}{:}\PY{n}{endDate}\PY{p}{]}
         \PY{n}{costs} \PY{o}{=} \PY{n}{df}\PY{p}{[}\PY{l+s+s1}{\PYZsq{}}\PY{l+s+s1}{Cost (millions 2013US\PYZdl{})}\PY{l+s+s1}{\PYZsq{}}\PY{p}{]}
\end{Verbatim}


    \begin{Verbatim}[commandchars=\\\{\}]
{\color{incolor}In [{\color{incolor}35}]:} \PY{n}{n}\PY{o}{=}\PY{n+nb}{len}\PY{p}{(}\PY{n}{df}\PY{p}{)}
         \PY{n}{n}
\end{Verbatim}


\begin{Verbatim}[commandchars=\\\{\}]
{\color{outcolor}Out[{\color{outcolor}35}]:} 55
\end{Verbatim}
            
    \begin{center}\rule{0.5\linewidth}{\linethickness}\end{center}

\paragraph{Ex 1.2 Normal QQ-plots}\label{ex-1.2-normal-qq-plots}

\begin{quote}
partie a
\end{quote}

    \(X\) suit le loi normale de moyenne \(\mu\) et de variance
\(\sigma^2\), donc \[ X \sim N (\mu, \sigma^2)\] Posons \$ Z =
\frac{X - \mu}{\sqrt{\sigma^2}}\$ une variable aléatoire qui suit la loi
Normale (Gaussienne) centrale, et reduite.donc,
\[ Z = \frac{X - \mu}{\sqrt{\sigma^2}} \sim N (0, 1)\] On deduit que
\[ X = \mu + \sqrt{\sigma^2} . Z\] Or, \$ X \sim N (\mu, \sigma\^{}2)\$
alors sa \(p\)-quantile s'écrit ( pour \(p \in ]0,1[\) ):
\[ F^{-1}(p;\mu, \sigma^2) = \mu + \sigma.\Phi^{-1}(p)\] et \$ Z \sim N
(0, 1)\$ alors sa \(p\)-quantile s'écrit:
\[ F^{-1}(p;0, 1) = 0 + 1.\Phi^{-1}(p) = \Phi^{-1}(p)\] On obtient donc:
\[F^{-1}(p;\mu, \sigma^2) = \mu + \sqrt{\sigma^2} . F^{-1}(p;0, 1)\]
CQFD

    \begin{quote}
partie b
\end{quote}

    \begin{Verbatim}[commandchars=\\\{\}]
{\color{incolor}In [{\color{incolor}36}]:} \PY{k+kn}{from} \PY{n+nn}{scipy} \PY{k}{import} \PY{n}{stats}
         \PY{k+kn}{import} \PY{n+nn}{matplotlib}\PY{n+nn}{.}\PY{n+nn}{pyplot} \PY{k}{as} \PY{n+nn}{plt}
         \PY{k+kn}{import} \PY{n+nn}{seaborn} \PY{k}{as} \PY{n+nn}{sns}
         \PY{n}{sns}\PY{o}{.}\PY{n}{set}\PY{p}{(}\PY{n}{font\PYZus{}scale}\PY{o}{=}\PY{l+m+mf}{1.5}\PY{p}{,} \PY{n}{style}\PY{o}{=}\PY{l+s+s2}{\PYZdq{}}\PY{l+s+s2}{whitegrid}\PY{l+s+s2}{\PYZdq{}}\PY{p}{)}
         \PY{n}{resNorm} \PY{o}{=} \PY{n}{stats}\PY{o}{.}\PY{n}{probplot}\PY{p}{(}\PY{n}{costs}\PY{p}{,}\PY{n}{dist} \PY{o}{=} \PY{n}{stats}\PY{o}{.}\PY{n}{distributions}\PY{o}{.}\PY{n}{norm}\PY{p}{,} \PY{n}{plot}\PY{o}{=}\PY{n}{plt}\PY{p}{)}
\end{Verbatim}


    \begin{center}
    \adjustimage{max size={0.9\linewidth}{0.9\paperheight}}{output_11_0.png}
    \end{center}
    { \hspace*{\fill} \\}
    
    \begin{center}\rule{0.5\linewidth}{\linethickness}\end{center}

\paragraph{Ex 1.3 Exponential
QQ-plots}\label{ex-1.3-exponential-qq-plots}

    \begin{quote}
partie a
\end{quote}

    \(X\) suit le loi exponentielle à paramètre \(\lambda\),
\(X \sim \varepsilon\ (\lambda)\). Alors, sa densité de probabilité
sera:\[f(x;\lambda)=\lambda e^{-\lambda x},\ x\geq0\] D'où sa fonction
de répartition
:\[F(x;\lambda) = \int_0^x \lambda e^{-\lambda t}dt = 1 - e^{-\lambda.x} \]
En calculant l'inverse de \(F(x;\lambda)\), on obtient la \(p\)-quantile
de \(X\). Elle s'écrit :
\[y = 1 - e^{-\lambda.x}\\- e^{-\lambda.x}= y - 1 \\e^{-\lambda.x}=1-y\\-\lambda.x=\ln(1-y)\\x=-\frac{1}{\lambda}{\ln(1-y)}\]
En remplacant \(x\) par \(F^{-1}\) et \(y\) par \(p\) avex
\(p \in ]0,1[\) : \[F^{-1}(p;\lambda) = -\frac{1}{\lambda}{\ln(1-p)}\]
Pour une variable aleatoire qui suit la loi exponentielle normalisée:
\(Y \sim exp (1)\), sa \(p\)-quantile s'écrit:
\[F^{-1}(p;1) = -\frac{1}{1}{\ln(1-p)} = -\ln(1-p)\] On obtient donc:
\[F^{-1}(p;\lambda) = \frac{1}{\lambda}{F^{-1}(p;1)}\] CQFD

    \begin{quote}
partie b
\end{quote}

    \begin{Verbatim}[commandchars=\\\{\}]
{\color{incolor}In [{\color{incolor}37}]:} \PY{n}{resExpon} \PY{o}{=} \PY{n}{stats}\PY{o}{.}\PY{n}{probplot}\PY{p}{(}\PY{n}{costs}\PY{p}{,}\PY{n}{dist} \PY{o}{=} \PY{n}{stats}\PY{o}{.}\PY{n}{distributions}\PY{o}{.}\PY{n}{expon}\PY{p}{,} \PY{n}{plot}\PY{o}{=}\PY{n}{plt}\PY{p}{)}
\end{Verbatim}


    \begin{center}
    \adjustimage{max size={0.9\linewidth}{0.9\paperheight}}{output_16_0.png}
    \end{center}
    { \hspace*{\fill} \\}
    
    \begin{center}\rule{0.5\linewidth}{\linethickness}\end{center}

\paragraph{Ex 1.4 Conclusion}\label{ex-1.4-conclusion}

    \begin{Verbatim}[commandchars=\\\{\}]
{\color{incolor}In [{\color{incolor}38}]:} \PY{n}{f} \PY{o}{=} \PY{n}{plt}\PY{o}{.}\PY{n}{figure}\PY{p}{(}\PY{n}{figsize}\PY{o}{=}\PY{p}{(}\PY{l+m+mi}{20}\PY{p}{,}\PY{l+m+mi}{5}\PY{p}{)}\PY{p}{)}
         \PY{n}{ax1} \PY{o}{=} \PY{n}{f}\PY{o}{.}\PY{n}{add\PYZus{}subplot}\PY{p}{(}\PY{l+m+mi}{121}\PY{p}{)}
         \PY{n}{ax2} \PY{o}{=} \PY{n}{f}\PY{o}{.}\PY{n}{add\PYZus{}subplot}\PY{p}{(}\PY{l+m+mi}{122}\PY{p}{)}
         \PY{n}{resNorm} \PY{o}{=} \PY{n}{stats}\PY{o}{.}\PY{n}{probplot}\PY{p}{(}\PY{n}{costs}\PY{p}{,}\PY{n}{dist} \PY{o}{=} \PY{n}{stats}\PY{o}{.}\PY{n}{distributions}\PY{o}{.}\PY{n}{norm}\PY{p}{,} \PY{n}{plot}\PY{o}{=}\PY{n}{ax1}\PY{p}{)}
         \PY{n}{resExpon} \PY{o}{=} \PY{n}{stats}\PY{o}{.}\PY{n}{probplot}\PY{p}{(}\PY{n}{costs}\PY{p}{,}\PY{n}{dist} \PY{o}{=} \PY{n}{stats}\PY{o}{.}\PY{n}{distributions}\PY{o}{.}\PY{n}{expon}\PY{p}{,} \PY{n}{plot}\PY{o}{=}\PY{n}{ax2}\PY{p}{)}
\end{Verbatim}


    \begin{center}
    \adjustimage{max size={0.9\linewidth}{0.9\paperheight}}{output_18_0.png}
    \end{center}
    { \hspace*{\fill} \\}
    
    D'après la definition des QQ-plots et leurs utilités à comparer la
distribution de deux variables aleatoires en traçant leurs quantiles,
l'une en fonction de lautre (dans notre cas, la distribution théorique
contre celle obtenue actuellement), on peut déduire que la distribution
exponentielle semble être plus plausible que celle du modèle normale.

On peut aisement voir dans la première figure au-dessus que les points
ne sont pas assez lineaire, en indiquant un peu une exponentiation par
rapport à la droite représentant la loi Normale. (c'était alors
previsible que les coûts suivent une loi exponentielle)

Par contre, pour le diagramme Qentile-Qentile de la loi exponentielle
contre les coûts des accidents avant Three Mile Island, on peut remarqué
une colinearité entre les distributions, indiquant que les coûts sont
plus probablement distribués exponentiellement.

    \begin{center}\rule{0.5\linewidth}{\linethickness}\end{center}

\subsubsection{\texorpdfstring{\[EXERCICE\ 2\]}{EXERCICE\textbackslash{} 2}}\label{exercice-2}

    \begin{center}\rule{0.5\linewidth}{\linethickness}\end{center}

\subsubsection{Ex 2.1}\label{ex-2.1}

    D'après la donnée, \(\mathbb{P}_\lambda(X_1>x) = \exp^{-\lambda x}\)
pour \(x\geq0\), et 1 sinon.

Alors la fonction de repartition de \(X_1\) sera :
\[F(X_1) = \mathbb{P}_\lambda(X_1\leq x) = 1 - \exp^{-\lambda x}\]
Alors, on calcule la densité de probabilité
\(\mathbb{P}(X_1,\lambda)=\ f(X_1,\lambda)\) en dérivant par rapport à
X\_1:
\[  f(X_1,\lambda) =\ \frac{\partial F(X_1)}{\partial X_1} = \frac{\partial (1 - \exp^{-\lambda X_1})}{\partial X_1}=\lambda \exp^{-\lambda X_1}\]
Or on a considéré les accidents indépendants et identiquements
distribués, alors:
\[\mathbb{P}(X,\lambda) = \prod_{i=1}^n\mathbb{P}(X_i,\lambda)\] Donc,
la densité de probabilité du vecteur \(X\) est:
\[\mathbb{P}(X,\lambda) = \lambda^n \exp^{-\lambda \sum_{i=1}^nX_i}\]

\begin{center}\rule{0.5\linewidth}{\linethickness}\end{center}

    Pour trouver l'estimateur de maximum vraisemblance de \(\lambda\), on
pose l'estimateur \(\widehat{\theta}_{MV}(X)= \widehat{\lambda}_n\) qui
verifie la relation:
\[\widehat{\theta}_{MV}(X) = \arg \min_{t \in \Theta}{M(X,t)}\] où
\(M(X,t)\) est un contraste donné par :
\[M(X,t)=- \log{\mathbb{P}(X,t)} = -\log{f(X,t)}\] Donc, \$
\widehat{\lambda}\_n\$ sera la valeur qui minimise \(M(X,t)\).

Resolvons l'equation \(\frac{\partial M(X,t)}{\partial t} = 0\)
\[\frac{\partial (-\log{\mathbb{P}(X,t))}}{\partial t} = 0\]
\[\frac{\partial [-\log{(t^n\ \exp^{-t \sum_{i=1}^n X_i} })]}{\partial t} = 0\]
\[\frac{n t^{n-1}\exp^{-t \sum_{i=1}^n X_i} - t^n \sum_{i=1}^n X_i \exp^{-t \sum_{i=1}^n X_i}}
{t^n\ \exp^{-t \sum_{i=1}^n X_i}} 
= 0 \] En simplifiant, on obtient: \[n t^{-1} -\sum_{i=1}^n X_i=0\] D'où
l'estimateur de max vraisemblance sera:
\[  \widehat{\lambda}_n=\widehat{t}=\frac{n}{\sum_{i=1}^n X_i}\]

    \begin{center}\rule{0.5\linewidth}{\linethickness}\end{center}

\subsubsection{Ex 2.2}\label{ex-2.2}

    Calcule de \(\lambda_n\) :

    \begin{Verbatim}[commandchars=\\\{\}]
{\color{incolor}In [{\color{incolor}39}]:} \PY{k+kn}{import} \PY{n+nn}{numpy} \PY{k}{as} \PY{n+nn}{np}
         \PY{n}{totalCases} \PY{o}{=} \PY{n+nb}{len}\PY{p}{(}\PY{n}{df}\PY{p}{)} \PY{c+c1}{\PYZsh{}n}
         \PY{n}{totalCosts} \PY{o}{=} \PY{n}{costs}
         \PY{n}{lambdaN} \PY{o}{=} \PY{n+nb}{len}\PY{p}{(}\PY{n}{costs}\PY{p}{)}\PY{o}{/}\PY{n}{np}\PY{o}{.}\PY{n}{sum}\PY{p}{(}\PY{n}{totalCosts}\PY{p}{)}
         \PY{n+nb}{print}\PY{p}{(}\PY{l+s+s2}{\PYZdq{}}\PY{l+s+s2}{La valeure calculée de \PYZdl{}lambda\PYZus{}n\PYZdl{} est}\PY{l+s+s2}{\PYZdq{}}\PY{p}{,}\PY{n}{lambdaN}\PY{p}{)}
\end{Verbatim}


    \begin{Verbatim}[commandchars=\\\{\}]
La valeure calculée de \$lambda\_n\$ est 0.0013767725947156965

    \end{Verbatim}

    \begin{Verbatim}[commandchars=\\\{\}]
{\color{incolor}In [{\color{incolor}40}]:} \PY{k+kn}{from} \PY{n+nn}{scipy}\PY{n+nn}{.}\PY{n+nn}{stats} \PY{k}{import} \PY{n}{expon}
         \PY{n}{fig}\PY{p}{,} \PY{n}{ax} \PY{o}{=} \PY{n}{plt}\PY{o}{.}\PY{n}{subplots}\PY{p}{(}\PY{l+m+mi}{1}\PY{p}{,} \PY{l+m+mi}{1}\PY{p}{)}
         \PY{n}{moyenne} \PY{o}{=} \PY{l+m+mi}{1}\PY{o}{/}\PY{n}{lambdaN}
         \PY{n}{x} \PY{o}{=} \PY{n}{np}\PY{o}{.}\PY{n}{linspace}\PY{p}{(}\PY{n}{expon}\PY{o}{.}\PY{n}{ppf}\PY{p}{(}\PY{l+m+mf}{0.01}\PY{p}{,}\PY{n}{scale}\PY{o}{=}\PY{n}{moyenne}\PY{p}{)}\PY{p}{,} \PY{n}{expon}\PY{o}{.}\PY{n}{ppf}\PY{p}{(}\PY{o}{.}\PY{l+m+mi}{99}\PY{p}{,}\PY{n}{scale}\PY{o}{=}\PY{n}{moyenne}\PY{p}{)}\PY{p}{,}\PY{l+m+mi}{100}\PY{p}{)}
         \PY{n}{ax}\PY{o}{.}\PY{n}{plot}\PY{p}{(}\PY{n}{x}\PY{p}{,} \PY{n}{expon}\PY{o}{.}\PY{n}{pdf}\PY{p}{(}\PY{n}{x}\PY{p}{,}\PY{l+m+mi}{0}\PY{p}{,}\PY{n}{moyenne}\PY{p}{)}\PY{p}{,}\PY{l+s+s1}{\PYZsq{}}\PY{l+s+s1}{r\PYZhy{}}\PY{l+s+s1}{\PYZsq{}}\PY{p}{,} \PY{n}{lw}\PY{o}{=}\PY{l+m+mi}{5}\PY{p}{,} \PY{n}{alpha}\PY{o}{=}\PY{l+m+mf}{0.6}\PY{p}{,} \PY{n}{label}\PY{o}{=}\PY{l+s+s1}{\PYZsq{}}\PY{l+s+s1}{Densité de la loi exp à lambda\PYZus{}n}\PY{l+s+s1}{\PYZsq{}}\PY{p}{)}
         \PY{n}{entries}\PY{p}{,} \PY{n}{bin\PYZus{}edges}\PY{p}{,} \PY{n}{c}\PY{o}{=}\PY{n}{ax}\PY{o}{.}\PY{n}{hist}\PY{p}{(}\PY{n}{costs}\PY{p}{,} \PY{n}{density}\PY{o}{=}\PY{k+kc}{True}\PY{p}{,} \PY{n}{histtype}\PY{o}{=}\PY{l+s+s1}{\PYZsq{}}\PY{l+s+s1}{stepfilled}\PY{l+s+s1}{\PYZsq{}}\PY{p}{,} 
                                       \PY{n}{alpha}\PY{o}{=}\PY{l+m+mf}{0.8}\PY{p}{,}\PY{n}{label}\PY{o}{=}\PY{l+s+s1}{\PYZsq{}}\PY{l+s+s1}{Cout (millions 2013US\PYZdl{})}\PY{l+s+s1}{\PYZsq{}}\PY{p}{,}
                                      \PY{n}{bins} \PY{o}{=} \PY{l+m+mi}{17}\PY{p}{)}
         \PY{c+c1}{\PYZsh{}print(sum(entries * np.diff(bin\PYZus{}edges)))}
         \PY{n}{ax}\PY{o}{.}\PY{n}{legend}\PY{p}{(}\PY{n}{loc}\PY{o}{=}\PY{l+s+s1}{\PYZsq{}}\PY{l+s+s1}{best}\PY{l+s+s1}{\PYZsq{}}\PY{p}{,} \PY{n}{frameon}\PY{o}{=}\PY{k+kc}{True}\PY{p}{)}
         \PY{n}{plt}\PY{o}{.}\PY{n}{show}\PY{p}{(}\PY{p}{)}
\end{Verbatim}


    \begin{center}
    \adjustimage{max size={0.9\linewidth}{0.9\paperheight}}{output_27_0.png}
    \end{center}
    { \hspace*{\fill} \\}
    
    \begin{center}\rule{0.5\linewidth}{\linethickness}\end{center}

\paragraph{Ex 2.3}\label{ex-2.3}

    Le modèle est regulier. Grandeur d'intérêt :
\(g_1(\lambda)=\frac{1}{\lambda}\); Statistique d'étude :
\(T_1(X)= \frac{1}{n}\sum_{i=1}^nX_i\)

Pour que \(T_1(X)\) soit un estimateur efficace pour \(g_1(\lambda)\),
il doit verifier ces deux conditions: \textgreater{}1- \(T_1(X)\) est
sans biais ==\textgreater{} \(b(g_1,T_1) = 0\)

\begin{quote}
2- la variance de \(T_1(X)\) doit atteindre la borne de Cramer-Rao
\end{quote}

    1- Pour la première condition: \[
    b(g_1,T_1) = \mathbb{E}_\lambda[T_1(X)-g_1(\lambda)] = \mathbb{E}_\lambda[\frac{1}{n}\sum_{i=1}^nX_i - \frac{1}{\lambda}]\\
    b(g_1,T_1) = \frac{1}{n}\mathbb{E}_\lambda[\sum_{i=1}^nX_i] - \frac{1}{\lambda}\\
    b(g_1,T_1) = \frac{1}{n}\sum_{i=1}^n\mathbb{E}_\lambda[X_i] - \frac{1}{\lambda}\\
\] Or, les les accidents indépendants et identiquements distribués,
alors \(X_i \stackrel{i.i.d.}{\sim} \varepsilon\ (\lambda)\), et
\(\mathbb{E}_\lambda[X_i] = 1/\lambda\), d'où : \[
     b(g_1,T_1) = \frac{1}{n}\sum_{i=1}^n\frac{1}{\lambda} - \frac{1}{\lambda} = 0
\]

    2- Pour la deuxième condition, on doit calculé la borne de Cramer-Rao,
qui est une borne inférieure de la variance de l'estimateur \(T_1\), cad
\(\mathbb{V}ar_\lambda(T_1) \geq borneCR\).

Au debut,
\[borneCR = \frac{[g_1'(\lambda)]^2}{I(\lambda)} = \frac{[-1/\lambda^2]^2}{I(\lambda)} = \frac{\frac{1}{\lambda^4}}{I(\lambda)}\]

où \(I(\lambda)\) est l'information de Fisher contenu dans la variable
d'étude. Dans notre cas c'est le vercteur \(X\) de dimension \(n\). Or,
l'information de Fisher est additive, donc \$I(\lambda) = n
I\_1(\lambda) \$

Elle est donnée par:
\(I(\lambda) := \mathbb{E}_\lambda[\ (\frac{\partial \log{\mathbb{P}(X,\lambda)}} {\partial\lambda})^2\ ]\)

Or, \(\mathbb{P}(X,\lambda)\) est infiniment differentiable (il suffit
deux) par rapport à \(\lambda\) (loi exponentielle), alors : \[
I_1(\lambda) = - \mathbb{E}_\lambda[\ \frac{\partial^2 \log{\mathbb{P}(x,\lambda)}} {\partial\lambda^2}]\\
    I_1(\lambda) = - \mathbb{E}_\lambda[\ \frac{\partial^2 \log(\lambda\exp^{-\lambda x})} {\partial\lambda^2}]= - \mathbb{E}_\lambda[\ \frac{\partial^2 (\log\lambda-\lambda x)} {\partial\lambda^2}]\\
            I_1(\lambda) = - \mathbb{E}_\lambda[\ \frac{\partial (1/\lambda - x)} {\partial\lambda}]= - \mathbb{E}_\lambda[-1/\lambda^2 - 0]\\
                I_1(\lambda) = \frac{1}{\lambda^2}
\] Donc, \$ I(\lambda)= \frac{n}{\lambda^2}\$

D'où la borne de Cramer Rao sera:
\[borneCR = \frac{\frac{1}{\lambda^4}}{\frac{n}{\lambda^2}} = \frac{1}{n\lambda^2}\]

Cherchons la variance de l'estimateur \(g_1(\lambda)\):

\[\mathbb{V}ar_\lambda(T_1) = \mathbb{V}ar_\lambda(\frac{1}{n}\sum_{i=1}^nX_i) = \frac{1}{n^2}\mathbb{V}ar_\lambda(\sum_{i=1}^nX_i)\]

Or, les les accidents indépendants et identiquements distribués, alors :
\(\mathbb{V}ar_\lambda(\sum_{i=1}^nX_i)= \sum_{i=1}^n\mathbb{V}ar_\lambda [X_i]\).
Donc,

\[
\mathbb{V}ar_\lambda(T_1) = \frac{1}{n^2}\sum_{i=1}^n\mathbb{V}ar_\lambda [X_i] = \frac{1}{n^2}\sum_{i=1}^n\frac{1}{\lambda^2} =
\frac{1}{n^2}\frac{n}{\lambda^2}\\ \mathbb{V}ar_\lambda(T_1) = \frac{1}{n\lambda^2} = borneCR
\]

Donc, \(T_1(X)\) vérifie les deux conditions d'efficacité et il est
efficace.

    \begin{center}\rule{0.5\linewidth}{\linethickness}\end{center}

\paragraph{Ex 2.4}\label{ex-2.4}

    Calcul de \(g_1\):

    \begin{Verbatim}[commandchars=\\\{\}]
{\color{incolor}In [{\color{incolor}41}]:} \PY{n}{T1} \PY{o}{=} \PY{n}{np}\PY{o}{.}\PY{n}{sum}\PY{p}{(}\PY{n}{totalCosts}\PY{p}{)}\PY{o}{/}\PY{n}{totalCases}
         \PY{n+nb}{print}\PY{p}{(}\PY{l+s+s2}{\PYZdq{}}\PY{l+s+s2}{L}\PY{l+s+s2}{\PYZsq{}}\PY{l+s+s2}{estimation de g1 par T1 sur l}\PY{l+s+s2}{\PYZsq{}}\PY{l+s+s2}{échantillon donné est}\PY{l+s+s2}{\PYZdq{}}\PY{p}{,}\PY{l+s+s2}{\PYZdq{}}\PY{l+s+si}{\PYZob{}:.3f\PYZcb{}}\PY{l+s+s2}{\PYZdq{}}\PY{o}{.}\PY{n}{format}\PY{p}{(}\PY{n}{T1}\PY{p}{)}\PY{p}{,}\PY{l+s+s2}{\PYZdq{}}\PY{l+s+s2}{million\PYZdl{}}\PY{l+s+s2}{\PYZdq{}}\PY{p}{)}
\end{Verbatim}


    \begin{Verbatim}[commandchars=\\\{\}]
L'estimation de g1 par T1 sur l'échantillon donné est 726.336 million\$

    \end{Verbatim}

    \begin{center}\rule{0.5\linewidth}{\linethickness}\end{center}

\paragraph{Ex 2.5}\label{ex-2.5}

    Le nouveau estimateur est: \$\tilde{T}\_\{1,\eta\}(X) = \eta T\_1(X) \$

Cherchons son risque quadratique:
\(R(\lambda,\tilde{T}_{1,\eta}) = b^2_\lambda(\tilde{T}_{1,\eta}) +\mathbb{V}ar_\lambda(\tilde{T}_{1,\eta})\)

tel que
\[b_\lambda(g_1,\tilde{T}_{1,\eta}) = \mathbb{E}_\lambda[\tilde{T}_{1,\eta}-g_1(\lambda)] = \mathbb{E}_\lambda[\frac{\eta}{n}\sum_{i=1}^nX_i - \frac{1}{\lambda}]\\
    b(g_1,\tilde{T}_{1,\eta}) = \frac{\eta}{n}\mathbb{E}_\lambda[\sum_{i=1}^nX_i] - \frac{1}{\lambda}= \frac{\eta}{n}\sum_{i=1}^n\mathbb{E}_\lambda[X_i] - \frac{1}{\lambda}\\
    b(g_1,\tilde{T}_{1,\eta}) = \frac{\eta}{n}\sum_{i=1}^n\frac{1}{\lambda} - \frac{1}{\lambda}= \frac{\eta}{\lambda} - \frac{1}{\lambda}\\
    b(g_1,\tilde{T}_{1,\eta}) = \frac{\eta-1}{\lambda}
\]

et que

\[\mathbb{V}ar_\lambda(\tilde{T}_{1,\eta}) = \mathbb{V}ar_\lambda(\frac{\eta}{n}\sum_{i=1}^nX_i) = \frac{\eta^2}{n^2}\mathbb{V}ar_\lambda(\sum_{i=1}^nX_i)= \frac{\eta^2}{n^2}\sum_{i=1}^n\frac{1}{\lambda^2} =
\frac{\eta^2}{n^2}\frac{n}{\lambda^2}\\ 
\mathbb{V}ar_\lambda(\tilde{T}_{1,\eta}) = \frac{\eta^2}{n\lambda^2}\]

d'où

\[R(\lambda,\tilde{T}_{1,\eta}) = (\frac{\eta-1}{\lambda})^2 + \frac{\eta^2}{n\lambda^2} 
                                = \frac{n(\eta^2-2 \eta +1)}{n\lambda^2} + \frac{\eta^2}{n\lambda^2}\\
R(\lambda,\tilde{T}_{1,\eta})   =\frac{(n+1)\eta^2-2 n\eta +n}{n\lambda^2} \]

Pour que le nouveau risque soit inférieur à l'antécedent, il faut
résoudre l'inéquation suivante:
\[(n+1)\eta^2-2 n\eta +n < 1 ==> (n+1)\eta^2-2 n\eta + n-1 < 0\] qui
possède comme solution: \[ \frac{n-1}{n+1}<\eta<1\] Pour notre
echantillon: \[ \frac{55-1}{55+1}<\eta<1 \] Donc,
\[\frac{27}{28}<\eta<1\]

Ce résultat n'est pas en contradiction avec la question précédente
puisque cet estimateur n'est pas efficace, il possède un biais non nul.
C'est à dire qu'il y a une certaine region (celle retrouvée de
\(\eta\)), où cet estimateur est meilleur que celui non pondéré par
\(\eta\), mais son risque serait mauvais en dehors de cette région
(graphique indicative enfin de l'exercice 2).

    \begin{center}\rule{0.5\linewidth}{\linethickness}\end{center}

\paragraph{Ex 2.6}\label{ex-2.6}

    Soit \(m_\lambda = mediane(X)\) tel que
\(\int_{-\infty}^{m_\lambda}\lambda\exp^{-\lambda x}.dx = 50\%\)
==\textgreater{}
\(\int_{0}^{m_\lambda}\lambda\exp^{-\lambda x}.dx = 50\%\)

\$ - \exp\^{}\{-\lambda x\} \textbar{}\^{}m\_0 = 1/2 ==\textgreater{}
\exp\textsuperscript{\{-\lambda .0\}-\exp}\{-\lambda .m\} =
1/2\textbackslash{} 1-\exp\^{}\{-\lambda .m\} = 1/2 ==\textgreater{}
\exp\^{}\{-\lambda .m\} = 1/2\textbackslash{} -\lambda .m = -ln(2)\$

Donc, \[m_\lambda = \frac{\ln(2)}{\lambda}\] Pour que
\(\tilde{T}_{1,\eta}(X)\) soit l'estimateur sans biais de la mediane, il
faut que le bias : \(b_\lambda(m_\lambda,\tilde{T}_{1,\eta})\) soit nul.

\[b_\lambda(m_\lambda,\tilde{T}_{1,\eta}) = \mathbb{E}_\lambda[\tilde{T}_{1,\eta}-m(\lambda)] = \frac{\eta}{n}\mathbb{E}_\lambda[\sum_{i=1}^nX_i] - \frac{\ln(2)}{\lambda} = \frac{\eta}{\lambda}- \frac{\ln(2)}{\lambda} = 0\]
\[\eta - \ln(2) = 0\ ,\ \forall \lambda \neq 0\] Donc,\[\eta = \ln(2)\]

    Calcule de la valeure de l'estimateur \(\tilde{T}_{1,\eta}(X)\) pour
\(\eta = \ln{2}\) sur l'échantillon donné :

    \begin{Verbatim}[commandchars=\\\{\}]
{\color{incolor}In [{\color{incolor}42}]:} \PY{k+kn}{import} \PY{n+nn}{math}
         \PY{n}{T1\PYZus{}tilde} \PY{o}{=} \PY{n}{math}\PY{o}{.}\PY{n}{log}\PY{p}{(}\PY{l+m+mi}{2}\PY{p}{)}\PY{o}{*}\PY{n}{np}\PY{o}{.}\PY{n}{sum}\PY{p}{(}\PY{n}{totalCosts}\PY{p}{)}\PY{o}{/}\PY{n}{totalCases}
         \PY{n}{T1\PYZus{}tilde} \PY{o}{=} \PY{n}{math}\PY{o}{.}\PY{n}{floor}\PY{p}{(}\PY{n}{T1\PYZus{}tilde}\PY{o}{*}\PY{l+m+mi}{100}\PY{p}{)}\PY{o}{/}\PY{l+m+mi}{100}
         \PY{n+nb}{print}\PY{p}{(}\PY{l+s+s2}{\PYZdq{}}\PY{l+s+s2}{La médiane estimée de cette echantillon est}\PY{l+s+s2}{\PYZdq{}}\PY{p}{,} \PY{n}{f}\PY{l+s+s2}{\PYZdq{}}\PY{l+s+si}{\PYZob{}T1\PYZus{}tilde:,\PYZcb{}}\PY{l+s+s2}{\PYZdq{}}\PY{p}{,}\PY{l+s+s2}{\PYZdq{}}\PY{l+s+s2}{millions \PYZdl{}}\PY{l+s+s2}{\PYZdq{}}\PY{p}{)}
\end{Verbatim}


    \begin{Verbatim}[commandchars=\\\{\}]
La médiane estimée de cette echantillon est 503.45 millions \$

    \end{Verbatim}

    Calcule de la médiane empirique:

    \begin{Verbatim}[commandchars=\\\{\}]
{\color{incolor}In [{\color{incolor}43}]:} \PY{n}{median} \PY{o}{=} \PY{n+nb}{sorted}\PY{p}{(}\PY{n}{totalCosts}\PY{p}{)}\PY{p}{[}\PY{n}{n}\PY{o}{/}\PY{o}{/}\PY{l+m+mi}{2}\PY{p}{]}
         \PY{n+nb}{print}\PY{p}{(}\PY{l+s+s2}{\PYZdq{}}\PY{l+s+s2}{La médiane empirique de cette echantillon est}\PY{l+s+s2}{\PYZdq{}}\PY{p}{,} \PY{n}{f}\PY{l+s+s2}{\PYZdq{}}\PY{l+s+si}{\PYZob{}median:,\PYZcb{}}\PY{l+s+s2}{\PYZdq{}}\PY{p}{,}\PY{l+s+s2}{\PYZdq{}}\PY{l+s+s2}{millions \PYZdl{}}\PY{l+s+s2}{\PYZdq{}}\PY{p}{)}
\end{Verbatim}


    \begin{Verbatim}[commandchars=\\\{\}]
La médiane empirique de cette echantillon est 89.9 millions \$

    \end{Verbatim}

    \textbf{Comparaison}

La moyenne empirique est désormais plus petite que celle estimée, ce qui
est possible puisque l'echantillon de travail (n=55) n'est pas très
longue, et il se peut qu'on ne voit pas la convergence. Donc, ces deux
médianes seront plus proches plus que n augmente.

    \begin{center}\rule{0.5\linewidth}{\linethickness}\end{center}

\paragraph{Ex 2.7}\label{ex-2.7}

    Calcul du risque quadratique de \(T_1(X)\):
\(R(\lambda, T_1) = EQM(\lambda, T_1) = \mathbb{V}ar_\lambda(T_1) = \frac{1}{n\lambda^2}\)

    \begin{Verbatim}[commandchars=\\\{\}]
{\color{incolor}In [{\color{incolor}44}]:} \PY{n}{risqueT1} \PY{o}{=} \PY{l+m+mi}{1}\PY{o}{/}\PY{p}{(}\PY{n}{n}\PY{o}{*}\PY{n}{lambdaN}\PY{o}{*}\PY{o}{*}\PY{l+m+mi}{2}\PY{p}{)}
         \PY{n+nb}{print}\PY{p}{(}\PY{l+s+s2}{\PYZdq{}}\PY{l+s+s2}{le risque quadratique sous T1 est}\PY{l+s+s2}{\PYZdq{}}\PY{p}{,}\PY{n}{f}\PY{l+s+s2}{\PYZdq{}}\PY{l+s+si}{\PYZob{}risqueT1:,\PYZcb{}}\PY{l+s+s2}{\PYZdq{}}\PY{p}{)}
\end{Verbatim}


    \begin{Verbatim}[commandchars=\\\{\}]
le risque quadratique sous T1 est 9,592.082057099926

    \end{Verbatim}

    Calcul du risque quadratique de \(\tilde{T}_{1,\eta}(X)\):

\(R(\lambda, \tilde{T}_{1,\eta}) = EQM(\lambda, \tilde{T}_{1,\eta}) = \mathbb{V}ar_\lambda(\tilde{T}_{1,\eta}) = \frac{(n+1)\eta^2 -2n\eta+n}{n\lambda^2}\)
pour \(\eta = \ln(2)\)

    \begin{Verbatim}[commandchars=\\\{\}]
{\color{incolor}In [{\color{incolor}45}]:} \PY{n}{eta} \PY{o}{=} \PY{n}{math}\PY{o}{.}\PY{n}{log}\PY{p}{(}\PY{l+m+mi}{2}\PY{p}{)}
         \PY{n}{risqueT1tilde} \PY{o}{=} \PY{p}{(}\PY{p}{(}\PY{n}{n}\PY{o}{+}\PY{l+m+mi}{1}\PY{p}{)}\PY{o}{*}\PY{n}{eta}\PY{o}{*}\PY{o}{*}\PY{l+m+mi}{2} \PY{o}{\PYZhy{}}\PY{l+m+mi}{2}\PY{o}{*}\PY{n}{n}\PY{o}{*}\PY{n}{eta} \PY{o}{+}\PY{n}{n}\PY{p}{)}\PY{o}{/}\PY{p}{(}\PY{n}{n}\PY{o}{*}\PY{n}{lambdaN}\PY{o}{*}\PY{o}{*}\PY{l+m+mi}{2}\PY{p}{)}
         \PY{n+nb}{print}\PY{p}{(}\PY{l+s+s2}{\PYZdq{}}\PY{l+s+s2}{le risque quadratique sous T1 tilde est}\PY{l+s+s2}{\PYZdq{}}\PY{p}{,}\PY{n}{f}\PY{l+s+s2}{\PYZdq{}}\PY{l+s+si}{\PYZob{}risqueT1tilde:,\PYZcb{}}\PY{l+s+s2}{\PYZdq{}}\PY{p}{)}
\end{Verbatim}


    \begin{Verbatim}[commandchars=\\\{\}]
le risque quadratique sous T1 tilde est 54,283.308555590236

    \end{Verbatim}

    \begin{Verbatim}[commandchars=\\\{\}]
{\color{incolor}In [{\color{incolor}46}]:} \PY{n+nb}{print}\PY{p}{(}\PY{l+s+s2}{\PYZdq{}}\PY{l+s+s2}{La difference est}\PY{l+s+s2}{\PYZdq{}}\PY{p}{,} \PY{n}{risqueT1tilde} \PY{o}{\PYZhy{}} \PY{n}{risqueT1}\PY{p}{)}
\end{Verbatim}


    \begin{Verbatim}[commandchars=\\\{\}]
La difference est 44691.22649849031

    \end{Verbatim}

    On note que le risque quadratique de \(T_1\) est meilleur que
\(\tilde{T}_1\) et cela est prévisible car \(\eta\) est hors du région
\$ 27/28 \textless{} \eta \textless{} 1\$

    \begin{Verbatim}[commandchars=\\\{\}]
{\color{incolor}In [{\color{incolor}47}]:} \PY{n+nb}{print}\PY{p}{(}\PY{n}{math}\PY{o}{.}\PY{n}{log}\PY{p}{(}\PY{l+m+mi}{2}\PY{p}{)}\PY{p}{,} \PY{l+s+s2}{\PYZdq{}}\PY{l+s+s2}{\PYZlt{}}\PY{l+s+s2}{\PYZdq{}}\PY{p}{,}\PY{l+m+mi}{27}\PY{o}{/}\PY{l+m+mi}{28}\PY{p}{)}
\end{Verbatim}


    \begin{Verbatim}[commandchars=\\\{\}]
0.6931471805599453 < 0.9642857142857143

    \end{Verbatim}

    La variation du risque quadratique de \(T_1(X)\) et de
\(\tilde{T}_{1,\eta}(X)\) en fonction de n sera:

    \begin{Verbatim}[commandchars=\\\{\}]
{\color{incolor}In [{\color{incolor}48}]:} \PY{n}{nVect} \PY{o}{=} \PY{n}{np}\PY{o}{.}\PY{n}{linspace}\PY{p}{(}\PY{l+m+mi}{1}\PY{p}{,}\PY{l+m+mi}{115}\PY{p}{,}\PY{l+m+mi}{1000}\PY{p}{)}
         \PY{n}{risqueT1VectN} \PY{o}{=} \PY{l+m+mi}{1}\PY{o}{/}\PY{p}{(}\PY{n}{nVect}\PY{o}{*}\PY{n}{lambdaN}\PY{o}{*}\PY{o}{*}\PY{l+m+mi}{2}\PY{p}{)}
         \PY{n}{risqueT1tildeVectN} \PY{o}{=} \PY{p}{(}\PY{p}{(}\PY{n}{nVect}\PY{o}{+}\PY{l+m+mi}{1}\PY{p}{)}\PY{o}{*}\PY{n}{eta}\PY{o}{*}\PY{o}{*}\PY{l+m+mi}{2} \PY{o}{\PYZhy{}}\PY{l+m+mi}{2}\PY{o}{*}\PY{n}{nVect}\PY{o}{*}\PY{n}{eta} \PY{o}{+}\PY{n}{nVect}\PY{p}{)}\PY{o}{/}\PY{p}{(}\PY{n}{nVect}\PY{o}{*}\PY{n}{lambdaN}\PY{o}{*}\PY{o}{*}\PY{l+m+mi}{2}\PY{p}{)}
         \PY{n}{rmin} \PY{o}{=} \PY{n}{np}\PY{o}{.}\PY{n}{min}\PY{p}{(}\PY{n}{risqueT1VectN}\PY{p}{)}
         \PY{n}{mask} \PY{o}{=} \PY{n}{np}\PY{o}{.}\PY{n}{array}\PY{p}{(}\PY{n}{risqueT1VectN}\PY{p}{)} \PY{o}{==} \PY{n}{rmin}
         \PY{n}{color} \PY{o}{=} \PY{n}{np}\PY{o}{.}\PY{n}{where}\PY{p}{(}\PY{n}{mask}\PY{p}{,} \PY{l+s+s1}{\PYZsq{}}\PY{l+s+s1}{yellow}\PY{l+s+s1}{\PYZsq{}}\PY{p}{,} \PY{l+s+s1}{\PYZsq{}}\PY{l+s+s1}{green}\PY{l+s+s1}{\PYZsq{}}\PY{p}{)}
         \PY{n}{rmin} \PY{o}{=} \PY{n}{np}\PY{o}{.}\PY{n}{min}\PY{p}{(}\PY{n}{risqueT1tildeVectN}\PY{p}{)}
         \PY{n}{mask} \PY{o}{=} \PY{n}{np}\PY{o}{.}\PY{n}{array}\PY{p}{(}\PY{n}{risqueT1tildeVectN}\PY{p}{)} \PY{o}{==} \PY{n}{rmin}
         \PY{n}{colorT} \PY{o}{=} \PY{n}{np}\PY{o}{.}\PY{n}{where}\PY{p}{(}\PY{n}{mask}\PY{p}{,} \PY{l+s+s1}{\PYZsq{}}\PY{l+s+s1}{red}\PY{l+s+s1}{\PYZsq{}}\PY{p}{,} \PY{l+s+s1}{\PYZsq{}}\PY{l+s+s1}{blue}\PY{l+s+s1}{\PYZsq{}}\PY{p}{)}
         \PY{n}{plt}\PY{o}{.}\PY{n}{ylabel}\PY{p}{(}\PY{l+s+s1}{\PYZsq{}}\PY{l+s+s1}{Risque}\PY{l+s+s1}{\PYZsq{}}\PY{p}{)}
         \PY{n}{plt}\PY{o}{.}\PY{n}{xlabel}\PY{p}{(}\PY{l+s+s1}{\PYZsq{}}\PY{l+s+s1}{Population n}\PY{l+s+s1}{\PYZsq{}}\PY{p}{)}
         \PY{n+nb}{print}\PY{p}{(}\PY{l+s+s2}{\PYZdq{}}\PY{l+s+s2}{Le risque est strictement décroissant pour n croissante.}\PY{l+s+s2}{\PYZdq{}}\PY{p}{)}
         \PY{n}{plt}\PY{o}{.}\PY{n}{plot}\PY{p}{(}\PY{n}{nVect}\PY{p}{,}\PY{n}{risqueT1tildeVectN}\PY{p}{,} \PY{n}{label} \PY{o}{=} \PY{l+s+s2}{\PYZdq{}}\PY{l+s+s2}{T1 tilde}\PY{l+s+s2}{\PYZdq{}}\PY{p}{)}
         \PY{n}{dessin}\PY{o}{=}\PY{n}{plt}\PY{o}{.}\PY{n}{plot}\PY{p}{(}\PY{n}{nVect}\PY{p}{,}\PY{n}{risqueT1VectN}\PY{p}{,}\PY{n}{label} \PY{o}{=} \PY{l+s+s2}{\PYZdq{}}\PY{l+s+s2}{T1}\PY{l+s+s2}{\PYZdq{}}\PY{p}{)}
         \PY{n}{ax}\PY{o}{=} \PY{n}{plt}\PY{o}{.}\PY{n}{gca}\PY{p}{(}\PY{p}{)}
         \PY{n}{ax}\PY{o}{=}\PY{n}{ax}\PY{o}{.}\PY{n}{legend}\PY{p}{(}\PY{n}{loc}\PY{o}{=}\PY{l+s+s1}{\PYZsq{}}\PY{l+s+s1}{best}\PY{l+s+s1}{\PYZsq{}}\PY{p}{,} \PY{n}{frameon}\PY{o}{=}\PY{k+kc}{True}\PY{p}{)}
\end{Verbatim}


    \begin{Verbatim}[commandchars=\\\{\}]
Le risque est strictement décroissant pour n croissante.

    \end{Verbatim}

    \begin{center}
    \adjustimage{max size={0.9\linewidth}{0.9\paperheight}}{output_53_1.png}
    \end{center}
    { \hspace*{\fill} \\}
    
    Donc \(T_1\) est meilleure que \(\tilde{T}_1\) pour les grandes
echantillons.

    EXTRA:

    Pour n fixe sur 55, on étudie la variation du rique en fonction de
\(\eta\):

    \begin{Verbatim}[commandchars=\\\{\}]
{\color{incolor}In [{\color{incolor}49}]:} \PY{n}{etaVect} \PY{o}{=} \PY{n}{np}\PY{o}{.}\PY{n}{linspace}\PY{p}{(}\PY{l+m+mf}{0.91}\PY{p}{,}\PY{l+m+mf}{1.05}\PY{p}{,}\PY{l+m+mi}{40}\PY{p}{)}
         \PY{n}{risqueVect} \PY{o}{=} \PY{p}{(}\PY{p}{(}\PY{l+m+mi}{55}\PY{o}{+}\PY{l+m+mi}{1}\PY{p}{)}\PY{o}{*}\PY{n}{etaVect}\PY{o}{*}\PY{o}{*}\PY{l+m+mi}{2} \PY{o}{\PYZhy{}}\PY{l+m+mi}{2}\PY{o}{*}\PY{l+m+mi}{55}\PY{o}{*}\PY{n}{etaVect} \PY{o}{+}\PY{l+m+mi}{55}\PY{p}{)}\PY{o}{/}\PY{p}{(}\PY{n}{n}\PY{o}{*}\PY{n}{lambdaN}\PY{o}{*}\PY{o}{*}\PY{l+m+mi}{2}\PY{p}{)}
         \PY{n}{rmin} \PY{o}{=} \PY{n}{np}\PY{o}{.}\PY{n}{min}\PY{p}{(}\PY{n}{risqueVect}\PY{p}{)}
         \PY{n}{mask} \PY{o}{=} \PY{n}{np}\PY{o}{.}\PY{n}{array}\PY{p}{(}\PY{n}{risqueVect}\PY{p}{)} \PY{o}{==} \PY{n}{rmin}
         \PY{n}{color} \PY{o}{=} \PY{n}{np}\PY{o}{.}\PY{n}{where}\PY{p}{(}\PY{n}{mask}\PY{p}{,} \PY{l+s+s1}{\PYZsq{}}\PY{l+s+s1}{red}\PY{l+s+s1}{\PYZsq{}}\PY{p}{,} \PY{l+s+s1}{\PYZsq{}}\PY{l+s+s1}{blue}\PY{l+s+s1}{\PYZsq{}}\PY{p}{)}
         \PY{n}{plt}\PY{o}{.}\PY{n}{ylabel}\PY{p}{(}\PY{l+s+s1}{\PYZsq{}}\PY{l+s+s1}{Risque}\PY{l+s+s1}{\PYZsq{}}\PY{p}{)}
         \PY{n}{plt}\PY{o}{.}\PY{n}{xlabel}\PY{p}{(}\PY{l+s+s1}{\PYZsq{}}\PY{l+s+s1}{\PYZdl{}}\PY{l+s+s1}{\PYZbs{}}\PY{l+s+s1}{eta\PYZdl{}}\PY{l+s+s1}{\PYZsq{}}\PY{p}{)}
         \PY{n+nb}{print}\PY{p}{(}\PY{l+s+s2}{\PYZdq{}}\PY{l+s+s2}{Le minmum risque est pour eta =}\PY{l+s+s2}{\PYZdq{}}\PY{p}{,}\PY{n}{etaVect}\PY{p}{[}\PY{n+nb}{int}\PY{p}{(}\PY{n}{np}\PY{o}{.}\PY{n}{where}\PY{p}{(}\PY{n}{risqueVect}\PY{o}{==}\PY{n}{rmin}\PY{p}{)}\PY{p}{[}\PY{l+m+mi}{0}\PY{p}{]}\PY{p}{)}\PY{p}{]}\PY{p}{)}
         \PY{n+nb}{print}\PY{p}{(}\PY{l+s+s2}{\PYZdq{}}\PY{l+s+s2}{Dans ce cas, le risque est}\PY{l+s+s2}{\PYZdq{}}\PY{p}{,}\PY{n}{rmin}\PY{p}{)}
         \PY{n}{dessin}\PY{o}{=}\PY{n}{plt}\PY{o}{.}\PY{n}{scatter}\PY{p}{(}\PY{n}{etaVect}\PY{p}{,}\PY{n}{risqueVect}\PY{p}{,}\PY{n}{color}\PY{o}{=}\PY{n}{color}\PY{p}{)}
         \PY{n}{plt}\PY{o}{.}\PY{n}{plot}\PY{p}{(}\PY{n}{etaVect}\PY{p}{,} \PY{p}{[}\PY{n}{risqueT1}\PY{p}{]}\PY{o}{*}\PY{n+nb}{len}\PY{p}{(}\PY{n}{etaVect}\PY{p}{)}\PY{p}{)}
\end{Verbatim}


    \begin{Verbatim}[commandchars=\\\{\}]
Le minmum risque est pour eta = 0.9817948717948718
Dans ce cas, le risque est 9420.859923843498

    \end{Verbatim}

\begin{Verbatim}[commandchars=\\\{\}]
{\color{outcolor}Out[{\color{outcolor}49}]:} [<matplotlib.lines.Line2D at 0x2175f9aabe0>]
\end{Verbatim}
            
    \begin{center}
    \adjustimage{max size={0.9\linewidth}{0.9\paperheight}}{output_57_2.png}
    \end{center}
    { \hspace*{\fill} \\}
    
    \begin{center}\rule{0.5\linewidth}{\linethickness}\end{center}

\subsubsection{\texorpdfstring{\[EXERCICE\ 3\]}{EXERCICE\textbackslash{} 3}}\label{exercice-3}

    \begin{center}\rule{0.5\linewidth}{\linethickness}\end{center}

\paragraph{Ex 3.1}\label{ex-3.1}

    Hypothèse nulle \(H_0\): le coût moyen d'un accident est inférieur à un
milliard de dollars

Hypothèse alternative \(H_1\): le coût moyen d'un accident est supérieur
ou égale à un milliard de dollars

Donc, \[H_0: 1/\lambda_0 < 1,000\] et \[H_1: 1/\lambda_1 \geq 1,000\]

en tenant compte que l'unité de coûts est le million 2013US\$

    \begin{center}\rule{0.5\linewidth}{\linethickness}\end{center}

\paragraph{Ex 3.2}\label{ex-3.2}

    En se basant sur le principe de Neyman-Pearson, on calcule le rapport de
vraisemblance résultant de notre hypothèse :

\[
   Z_{\lambda 0,\lambda 1}(X) = \frac{\mathbb{P}_{\lambda1}(X|H_1)}{\mathbb{P}_{\lambda0}(X|H_0)}
                              = \frac{f(X_{1:n})}{f(X_{1:n})}\\
                              = \frac{\prod_{i=1}^n f(X_i, \lambda_1)}{\prod_{i=1}^n f(X_i, \lambda_0)}\\
                              = \frac{\prod_{i=1}^n \lambda_1\exp{-\lambda_1 X_i}}{\prod_{i=1}^n \lambda_0\exp{-\lambda_0 X_i}}\\
                              =(\frac{\lambda_1}{\lambda_0})^n\frac{\exp^{-\sum_{i=1}^n \lambda_1 X_i}}{\exp^{-\sum_{i=1}^n \lambda_0 X_i}}\\
Z_{\lambda 0,\lambda 1}(X) = (\frac{\lambda_1}{\lambda_0})^n \exp^{-\sum_{i=1}^n (\lambda_1-\lambda_0) X_i}
\] Or, \(\lambda_1 < \lambda_0\), donc \(Z_{\lambda 0,\lambda 1}(X)\)
qui ne depend que du vecteur X, est strictement croissante.

Donc, le teste uniformément le plus puissant (U.P.P) est donné par
l'inégalité : \[ Z_{\lambda 0,\lambda 1}(X) > C \] où \(C\) est donnée
par l'égalité : \[ \mathbb{P}(Z_{\lambda 0,\lambda 1}(X) > C) = \alpha\]
où \(\alpha\) est le niveau de signification voulu du teste.

En utilisant la statistique de l'exercice \(T_1(X)\), on peut trouver la
région de rejet de ce test.

\textbf{Donc, on rejette \(H_0\) pour: \(T_1(X) > C'\)}
\[ \mathbb{P}( T_1(X) > C') = \alpha\\ \mathbb{P}( \frac{1}{n}\sum^n_{i=1}X_i > C') = \alpha\\
 \mathbb{P}( \frac{1}{n}\sum^n_{i=1}X_i \leq C') = 1 - \alpha\\
 \mathbb{P}( \sum^n_{i=1}X_i \leq nC') = 1 - \alpha
\] Donc, en posant \(q_{1-\alpha}\) le quantile de la loi de
distribution de la somme des \(Xi\), on obtient:

\[C' = \frac{q_{1-\alpha}}{n}\]

Or, la somme des exponentielles suit une loi \(\Gamma(k,\theta)\) car:

Soit \(X_{1:n}\) un vecteur de v.a. qui suient la loi exponentielle
\(\varepsilon(\lambda)\), donc
\[f(X_{1:n}) = \prod_{i=1}^n f(X^i)\ \ \ where\ \ \  X^i \sim \varepsilon\ (\lambda)\]
En fonctions caractéristiques et d'après la donnée du Mini-Projet, et
puisque \$\Phi\emph{\{X+Y\}(t)=\Phi}\{X\}(t).\Phi\_Y(t) \$ on peut
écrire:
\[ \Phi_{\sum X}(t) = \prod_{i=1}^n \frac{1}{1-\frac{it}{\lambda}} = \frac{1}{(1-\frac{it}{\lambda})^n}\]
En posant \(\theta = \frac{1}{\lambda}\), on obtient:
\[\Phi_X(t)= \frac{1}{(1-it\theta)^n}\] On déduit donc que
\(\sum^n_{i=1} X_i\) suit la loi gamma
\(\Gamma(k= n,\theta = 1/\lambda)\)

    \begin{center}\rule{0.5\linewidth}{\linethickness}\end{center}

\paragraph{Ex 3.3}\label{ex-3.3}

    Pour l'échantillon considéré, et pur \(\alpha = 0.05 = 5\%\):
\[ C' = \frac{q_{1-\alpha}^\Gamma}{n} = \frac{q^\Gamma_{0.95}}{55}\] On
utilisera la librairie stats de scipy pour trouver la valeur du quantile
(on peut de même utiliser la fonction de repartition fournie de la loi
Gamma).

    \begin{Verbatim}[commandchars=\\\{\}]
{\color{incolor}In [{\color{incolor}50}]:} \PY{n}{c}\PY{o}{=} \PY{n}{stats}\PY{o}{.}\PY{n}{gamma}\PY{o}{.}\PY{n}{ppf}\PY{p}{(}\PY{l+m+mf}{0.95}\PY{p}{,} \PY{n}{a}\PY{o}{=}\PY{n}{n}\PY{p}{,} \PY{n}{scale} \PY{o}{=} \PY{l+m+mi}{1000}\PY{p}{)}\PY{o}{/}\PY{l+m+mi}{55}
         \PY{n}{t1}\PY{o}{=} \PY{l+m+mi}{1}\PY{o}{/}\PY{n}{lambdaN}
         \PY{n+nb}{print}\PY{p}{(}\PY{l+s+s2}{\PYZdq{}}\PY{l+s+s2}{la valeur de C}\PY{l+s+s2}{\PYZsq{}}\PY{l+s+s2}{ est}\PY{l+s+s2}{\PYZdq{}}\PY{p}{,} \PY{n}{c}\PY{p}{)}
         \PY{n+nb}{print}\PY{p}{(}\PY{l+s+s2}{\PYZdq{}}\PY{l+s+s2}{la valeur de T1 sur notre échantillon est}\PY{l+s+s2}{\PYZdq{}}\PY{p}{,} \PY{n}{t1}\PY{p}{)}
\end{Verbatim}


    \begin{Verbatim}[commandchars=\\\{\}]
la valeur de C' est 1231.6379811669046
la valeur de T1 sur notre échantillon est 726.3363636363637

    \end{Verbatim}

    \begin{Verbatim}[commandchars=\\\{\}]
{\color{incolor}In [{\color{incolor}51}]:} \PY{k}{if} \PY{n}{t1} \PY{o}{\PYZgt{}} \PY{n}{c}\PY{p}{:}
             \PY{n+nb}{print}\PY{p}{(}\PY{l+s+s2}{\PYZdq{}}\PY{l+s+s2}{on rejette H0, notre hypothèse est falsifiée}\PY{l+s+s2}{\PYZdq{}}\PY{p}{)}
         \PY{k}{else}\PY{p}{:}
             \PY{n+nb}{print}\PY{p}{(}\PY{l+s+s2}{\PYZdq{}}\PY{l+s+s2}{on accepte H0, notre hypothèse est vérifiée}\PY{l+s+s2}{\PYZdq{}}\PY{p}{)}
\end{Verbatim}


    \begin{Verbatim}[commandchars=\\\{\}]
on accepte H0, notre hypothèse est vérifiée

    \end{Verbatim}

    Donc, on peut affirmer que le coût moyen d'accident est inférieur à un
milliard de dollars

    Pour la p-valeur, elle est égale à :
\[ P(T_1(X) < t_1 obtenue\ |\ H_0) \]

or \(t_1\) obtenue \(= 726.336\) millionsUS\$

donc, \[p\_valeur = P(T_1(X) <726.336 \ |\ H_0)\\
p\_valeur=P[\ T_1(X) <726.336 \ |\ T_1(X) \sim \Gamma(55,1000/55)\ ]\\ 
p\_valeur=\Gamma_{55,1000/55}(Z<726.336)
\]

    \begin{Verbatim}[commandchars=\\\{\}]
{\color{incolor}In [{\color{incolor}52}]:} \PY{n}{pvaleur}\PY{o}{=}\PY{n}{stats}\PY{o}{.}\PY{n}{gamma}\PY{o}{.}\PY{n}{pdf}\PY{p}{(}\PY{n}{t1}\PY{p}{,} \PY{n}{a} \PY{o}{=} \PY{n}{n}\PY{p}{,} \PY{n}{scale} \PY{o}{=} \PY{l+m+mi}{1000}\PY{o}{/}\PY{n}{n}\PY{p}{)}
         \PY{n+nb}{print}\PY{p}{(}\PY{l+s+s2}{\PYZdq{}}\PY{l+s+s2}{la p\PYZhy{}valeur est de l}\PY{l+s+s2}{\PYZsq{}}\PY{l+s+s2}{ordre de}\PY{l+s+s2}{\PYZdq{}}\PY{p}{,}\PY{l+s+s2}{\PYZdq{}}\PY{l+s+si}{\PYZob{}:.4\PYZcb{}}\PY{l+s+s2}{\PYZdq{}}\PY{o}{.}\PY{n}{format}\PY{p}{(}\PY{n}{pvaleur}\PY{o}{*}\PY{l+m+mi}{100}\PY{p}{)}\PY{p}{,}\PY{l+s+s2}{\PYZdq{}}\PY{l+s+s2}{\PYZpc{}}\PY{l+s+s2}{\PYZdq{}}\PY{p}{)}
         \PY{n+nb}{print}\PY{p}{(}\PY{l+s+s2}{\PYZdq{}}\PY{l+s+s2}{p\PYZhy{}valeur =}\PY{l+s+s2}{\PYZdq{}}\PY{p}{,}\PY{n}{pvaleur}\PY{p}{)}
\end{Verbatim}


    \begin{Verbatim}[commandchars=\\\{\}]
la p-valeur est de l'ordre de 0.03226 \%
p-valeur = 0.00032259476176334467

    \end{Verbatim}

    \begin{center}\rule{0.5\linewidth}{\linethickness}\end{center}

\paragraph{Ex 3.4}\label{ex-3.4}

    Comme on a juste démontré dans la question 3.2 que la somme des \(X_i\)
suit la loi Gamma tel que :
\[ \sum_{i=1}^n X_i \sim \Gamma(n, 1/\lambda)\] alors la statistique
\(T_1(X)= \frac{1}{n}\sum_{i=1}^nX_i\) est telle que:
\[ T_1 \sim \Gamma (k = n, \theta = \frac{1}{n\lambda})\] Dressons la
densité de probabilité de \(T_1(X)\) pour un moyen de coût des accidents
égale à 1 milliardsUS\$ = 1000 millionsUS\$, donc \(\lambda = 1/1000\):

    \begin{Verbatim}[commandchars=\\\{\}]
{\color{incolor}In [{\color{incolor}53}]:} \PY{n}{theta} \PY{o}{=} \PY{l+m+mi}{1}\PY{o}{/}\PY{l+m+mi}{1000}
         \PY{n}{x} \PY{o}{=} \PY{n}{np}\PY{o}{.}\PY{n}{linspace} \PY{p}{(}\PY{l+m+mf}{0.0005}\PY{p}{,} \PY{l+m+mf}{0.0015}\PY{p}{,} \PY{l+m+mi}{500}\PY{p}{)} 
         \PY{n}{y} \PY{o}{=} \PY{n}{stats}\PY{o}{.}\PY{n}{gamma}\PY{o}{.}\PY{n}{pdf}\PY{p}{(}\PY{n}{x}\PY{p}{,} \PY{n}{a}\PY{o}{=}\PY{n}{n}\PY{p}{,} \PY{n}{scale} \PY{o}{=} \PY{n}{theta}\PY{o}{/}\PY{n}{n}\PY{p}{)} 
         \PY{n}{a}\PY{o}{=}\PY{n}{plt}\PY{o}{.}\PY{n}{plot}\PY{p}{(}\PY{n}{x}\PY{p}{,} \PY{n}{y}\PY{p}{,} \PY{l+s+s2}{\PYZdq{}}\PY{l+s+s2}{y\PYZhy{}}\PY{l+s+s2}{\PYZdq{}}\PY{p}{,} \PY{n}{label}\PY{o}{=}\PY{p}{(}\PY{l+s+sa}{r}\PY{l+s+s1}{\PYZsq{}}\PY{l+s+s1}{\PYZdl{}}\PY{l+s+s1}{\PYZbs{}}\PY{l+s+s1}{alpha (k)=55, }\PY{l+s+s1}{\PYZbs{}}\PY{l+s+s1}{theta=1/1000[par million\PYZdl{}]\PYZdl{}}\PY{l+s+s1}{\PYZsq{}}\PY{p}{)}\PY{p}{)}
\end{Verbatim}


    \begin{center}
    \adjustimage{max size={0.9\linewidth}{0.9\paperheight}}{output_72_0.png}
    \end{center}
    { \hspace*{\fill} \\}
    
    Région de rejet:

Elle est donné pour les abscisse au delà des quelles la surface de la
distrubtion devient plus grande que 95\% de sa surface maximale.

    \begin{Verbatim}[commandchars=\\\{\}]
{\color{incolor}In [{\color{incolor}54}]:} \PY{n}{totalArea} \PY{o}{=} \PY{n}{np}\PY{o}{.}\PY{n}{trapz}\PY{p}{(}\PY{n}{y}\PY{p}{,} \PY{n}{dx}\PY{o}{=}\PY{l+m+mi}{5}\PY{p}{)}
         \PY{n}{area} \PY{o}{=} \PY{l+m+mi}{0}
         \PY{n}{i}\PY{o}{=}\PY{l+m+mi}{0}
         \PY{k}{while}\PY{p}{(}\PY{n}{area}\PY{o}{\PYZlt{}}\PY{n}{totalArea}\PY{o}{*}\PY{l+m+mf}{0.95} \PY{o+ow}{and} \PY{n}{i}\PY{o}{\PYZlt{}}\PY{n+nb}{len}\PY{p}{(}\PY{n}{y}\PY{p}{)}\PY{p}{)}\PY{p}{:}
             \PY{n}{area} \PY{o}{=}\PY{n}{np}\PY{o}{.}\PY{n}{trapz}\PY{p}{(}\PY{n}{y}\PY{p}{[}\PY{p}{:}\PY{n}{i}\PY{p}{]}\PY{p}{,} \PY{n}{dx}\PY{o}{=}\PY{l+m+mi}{5}\PY{p}{)}
             \PY{n}{i}\PY{o}{+}\PY{o}{=}\PY{l+m+mi}{1}
         
         \PY{n}{plt}\PY{o}{.}\PY{n}{plot}\PY{p}{(}\PY{n}{x}\PY{p}{[}\PY{p}{:}\PY{n}{i}\PY{p}{]}\PY{p}{,}\PY{n}{y}\PY{p}{[}\PY{p}{:}\PY{n}{i}\PY{p}{]}\PY{p}{,}\PY{n}{x}\PY{p}{[}\PY{n}{i}\PY{p}{:}\PY{p}{]}\PY{p}{,}\PY{n}{y}\PY{p}{[}\PY{n}{i}\PY{p}{:}\PY{p}{]}\PY{p}{)}
         \PY{n}{a}\PY{o}{=}\PY{n}{plt}\PY{o}{.}\PY{n}{fill\PYZus{}between}\PY{p}{(}\PY{n}{x}\PY{p}{[}\PY{n}{i}\PY{p}{:}\PY{p}{]}\PY{p}{,}\PY{n}{y}\PY{p}{[}\PY{n}{i}\PY{p}{:}\PY{p}{]}\PY{p}{,} \PY{n}{color} \PY{o}{=} \PY{l+s+s1}{\PYZsq{}}\PY{l+s+s1}{lime}\PY{l+s+s1}{\PYZsq{}}\PY{p}{)}
         \PY{n}{area}\PY{o}{/}\PY{n}{totalArea}
\end{Verbatim}


\begin{Verbatim}[commandchars=\\\{\}]
{\color{outcolor}Out[{\color{outcolor}54}]:} 0.9503953058868064
\end{Verbatim}
            
    \begin{center}
    \adjustimage{max size={0.9\linewidth}{0.9\paperheight}}{output_74_1.png}
    \end{center}
    { \hspace*{\fill} \\}
    
    \begin{center}\rule{0.5\linewidth}{\linethickness}\end{center}

\paragraph{Ex 3.5}\label{ex-3.5}

    La fonction de puissance d'un test est donnée par la probabilité que
l'échantillon soit rejeté par le test en tant que \(H_0\) est fausse,
celà est donné par:
\[Puissance = \mathbb{P}(T_1 \in \bar{A}\ |\ H_1) = 1 - \mathbb{P}(T_1 \in A\ |\ H_1)\]
Dans notre cas,
\[Puissance = 1 - \mathbb{P}(T_1(X) < C'\ |\ H_1)\ \ \ /\ \ \ C' = 1231.638\]

Or, sous \(H_1\), on a que \(\lambda \leq 1/1000\), donc pour la densité
de probabilité de \(T_1\), \(\theta\) varie sur l'intervale ouvert
{]}0,1000{[}

Domc,
\[Puissance (\lambda) = \Gamma_{n,\frac{1}{n\lambda}}((n,\lambda); Z < 1231.638)\]

    \begin{Verbatim}[commandchars=\\\{\}]
{\color{incolor}In [{\color{incolor}55}]:} \PY{n}{n\PYZus{}vect} \PY{o}{=} \PY{p}{[}\PY{l+m+mi}{10}\PY{p}{,}\PY{l+m+mi}{50}\PY{p}{,} \PY{l+m+mi}{100}\PY{p}{,} \PY{l+m+mi}{500}\PY{p}{,} \PY{l+m+mi}{100000}\PY{p}{]}
         \PY{n}{scale} \PY{o}{=} \PY{n}{np}\PY{o}{.}\PY{n}{linspace}\PY{p}{(}\PY{l+m+mf}{0.01}\PY{p}{,}\PY{l+m+mi}{2000}\PY{p}{,} \PY{l+m+mi}{100}\PY{p}{)}
         \PY{k}{for} \PY{n}{nb} \PY{o+ow}{in} \PY{n}{n\PYZus{}vect}\PY{p}{:}
             \PY{n}{y5} \PY{o}{=} \PY{l+m+mi}{1}\PY{o}{\PYZhy{}}\PY{n}{stats}\PY{o}{.}\PY{n}{gamma}\PY{o}{.}\PY{n}{cdf}\PY{p}{(}\PY{n}{c}\PY{p}{,} \PY{n}{a}\PY{o}{=}\PY{n}{nb}\PY{p}{,} \PY{n}{scale}\PY{o}{=}\PY{n}{scale}\PY{o}{/}\PY{n}{nb}\PY{p}{)}
             \PY{n}{plt}\PY{o}{.}\PY{n}{plot}\PY{p}{(}\PY{n}{scale}\PY{p}{,} \PY{n}{y5}\PY{p}{,} \PY{n}{label}\PY{o}{=}\PY{n}{nb}\PY{p}{)}
\end{Verbatim}


    \begin{center}
    \adjustimage{max size={0.9\linewidth}{0.9\paperheight}}{output_77_0.png}
    \end{center}
    { \hspace*{\fill} \\}
    
    On remarque que plus qu'on augmente la longueur de l'echantillon, plus
que la raideur de la fonction de puissance augmente, donc elle devient
plus puissante.

Les courbures indiquent que dans la zone de \(H_1\) cad \(1/\lambda\)
\textless{} 1000, la puissance de test est presque nulle, ce qui indique
des bonnes résultats auprès de ces tests.

    \begin{center}\rule{0.5\linewidth}{\linethickness}\end{center}

\paragraph{Ex 3.6}\label{ex-3.6}

    La statistique \(T_1\) possède comme espérance
\(\mathbb{E}(T_1,\lambda)=1/\lambda\) et variance
\(\mathbb{V}ar(T_1,\lambda)=\frac{1}{n.\lambda^2}\).

Or, le théorème central limite établit la convergence en loi de la somme
d'une suite de variables aléatoires ayant leur espérance et variance
connus, vers la loi normale.

D'où on peut assimiler la variable \(T_1\) une loi de distribution
Normale tel que:
\[ T_1 \sim N\ (\ \mu = \mathbb{E}(T_1,\lambda), \sigma^2 = \mathbb{V}ar(T_1,\lambda)\ )\]

Totalement écrite :
\[ T_1 \sim N (\mu = \frac{1}{\lambda}, \sigma = \frac{1}{\lambda\sqrt{n}})\]

Pour \(\lambda\) = 1 / 1000: (echelle par 1 million\$)

    \begin{Verbatim}[commandchars=\\\{\}]
{\color{incolor}In [{\color{incolor}56}]:} \PY{n}{mu} \PY{o}{=} \PY{l+m+mi}{1}\PY{o}{/}\PY{l+m+mi}{1000}
         \PY{n}{variance} \PY{o}{=} \PY{n}{mu}\PY{o}{*}\PY{o}{*}\PY{l+m+mi}{2}\PY{o}{/}\PY{l+m+mi}{55}
         \PY{n}{sigma} \PY{o}{=} \PY{n}{math}\PY{o}{.}\PY{n}{sqrt}\PY{p}{(}\PY{n}{variance}\PY{p}{)}
         \PY{n}{x} \PY{o}{=} \PY{n}{np}\PY{o}{.}\PY{n}{linspace}\PY{p}{(}\PY{n}{mu} \PY{o}{\PYZhy{}} \PY{l+m+mi}{5}\PY{o}{*}\PY{n}{sigma}\PY{p}{,} \PY{n}{mu} \PY{o}{+} \PY{l+m+mi}{5}\PY{o}{*}\PY{n}{sigma}\PY{p}{,} \PY{l+m+mi}{100}\PY{p}{)}
         \PY{n}{plt}\PY{o}{.}\PY{n}{plot}\PY{p}{(}\PY{n}{x}\PY{p}{,} \PY{n}{stats}\PY{o}{.}\PY{n}{norm}\PY{o}{.}\PY{n}{pdf}\PY{p}{(}\PY{n}{x}\PY{p}{,} \PY{n}{mu}\PY{p}{,} \PY{n}{sigma}\PY{p}{)}\PY{p}{)}
         \PY{n}{plt}\PY{o}{.}\PY{n}{show}\PY{p}{(}\PY{p}{)}
\end{Verbatim}


    \begin{center}
    \adjustimage{max size={0.9\linewidth}{0.9\paperheight}}{output_81_0.png}
    \end{center}
    { \hspace*{\fill} \\}
    
    \textbf{D'abord, on remarque la ressemblance entre les deux graphes de
lois Gamma et Normale pour \(T_1\)}

Test proposé : On adopte la même hypothèse de 3.1, et de la même façon
expliquée en 3.2-3, et en adoptant encore la statistique \(T_1(X)\), le
test revient à comparer: \[T_1(X) > C''\] où \(C''\) peut être calculer
comme suit:
\[ \mathbb{P}( T_1(X) > C'') = \alpha\\ \mathbb{P}(\frac{T_1(X)-\frac{1}{\lambda}}{\frac{1}{\sqrt{n}\lambda}} > k) = \alpha\ \ \ \ \ \ /\ \ \ \ k = \frac{C''-\frac{1}{\lambda}}{\frac{1}{\sqrt{n}\lambda}}
\] On remarque que la partie gauche est une loi Normale centralisée et
réduite, \$ Z \sim N(0,1)\$, alors: \[
 \mathbb{P}( Z \leq k) = 1 - \alpha\\
 \mathbb{P}( \sum^n_{i=1}X_i \leq nC') = 1 - \alpha
\] Donc, en posant \(q_{1-\alpha}^N\) le quantile de la loi \(N(0,1)\),
et pour \(\alpha = 0.05\), on obtient:

\[k = q_{0.95}^{N(0,1)}\]

    Résultats du test (programmatiquement):

    \begin{Verbatim}[commandchars=\\\{\}]
{\color{incolor}In [{\color{incolor}57}]:} \PY{n}{esp} \PY{o}{=} \PY{l+m+mi}{1000}
         \PY{n}{var} \PY{o}{=} \PY{n}{esp}\PY{o}{/}\PY{n}{math}\PY{o}{.}\PY{n}{sqrt}\PY{p}{(}\PY{n}{n}\PY{p}{)}
         \PY{n}{k}\PY{o}{=}\PY{n}{stats}\PY{o}{.}\PY{n}{norm}\PY{o}{.}\PY{n}{ppf}\PY{p}{(}\PY{l+m+mf}{0.95}\PY{p}{)}
         \PY{n}{cs}\PY{o}{=}\PY{n}{k}\PY{o}{*}\PY{n}{var} \PY{o}{+} \PY{n}{esp}
         \PY{k}{if} \PY{n}{t1} \PY{o}{\PYZgt{}} \PY{n}{cs}\PY{p}{:}
             \PY{n+nb}{print}\PY{p}{(}\PY{l+s+s2}{\PYZdq{}}\PY{l+s+s2}{on rejette H0, notre hypothèse est falsifiée}\PY{l+s+s2}{\PYZdq{}}\PY{p}{)}
         \PY{k}{else}\PY{p}{:}
             \PY{n+nb}{print}\PY{p}{(}\PY{l+s+s2}{\PYZdq{}}\PY{l+s+s2}{on accepte H0, notre hypothèse est vérifiée, car}\PY{l+s+s2}{\PYZdq{}}\PY{p}{,}\PY{n}{t1}\PY{p}{,}\PY{l+s+s2}{\PYZdq{}}\PY{l+s+s2}{\PYZlt{}}\PY{l+s+s2}{\PYZdq{}}\PY{p}{,}\PY{n}{c}\PY{p}{,}\PY{l+s+s2}{\PYZdq{}}\PY{l+s+se}{\PYZbs{}n}\PY{l+s+s2}{Donc, on accepte H\PYZus{}0}\PY{l+s+s2}{\PYZdq{}}\PY{p}{)}
\end{Verbatim}


    \begin{Verbatim}[commandchars=\\\{\}]
on accepte H0, notre hypothèse est vérifiée, car 726.3363636363637 < 1231.6379811669046 
Donc, on accepte H\_0

    \end{Verbatim}

    Donc, de même pour ce teste, on peut confirmer que le coût moyen des
accidents est inférieur à 1 milliards de dollars américaines.

    \section{FIN}\label{fin}


    % Add a bibliography block to the postdoc
    
    
    
    \end{document}
